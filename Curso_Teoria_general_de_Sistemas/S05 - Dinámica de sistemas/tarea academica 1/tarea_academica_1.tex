\documentclass[11pt,a4paper]{article}
\usepackage[spanish]{babel}
\usepackage[utf8]{inputenc}
\usepackage[T1]{fontenc}
\usepackage{geometry}
\usepackage{enumitem}
\usepackage{booktabs}
\usepackage{hyperref}
\usepackage{setspace}
\usepackage{graphicx}
\geometry{margin=2.5cm}
\setlist[itemize]{topsep=2pt,itemsep=2pt}
\setlist[enumerate]{topsep=2pt,itemsep=2pt}
\hypersetup{colorlinks=true,linkcolor=black,urlcolor=black}

\begin{document}

\begin{center}
{\LARGE \textbf{TAREA ACADÉMICA 1 (TA1) -- Teoría General de Sistemas}}\\[6pt]
{\large \textbf{Semana 5} \quad|\quad \textbf{Peso: 20\%}}\\[2pt]
\rule{0.95\linewidth}{0.4pt}
\end{center}

\vspace{4pt}
\noindent \textbf{Curso:} Teoría General de Sistemas (100000I51N) \hfill \textbf{Ciclo:} 2025-2\\
\noindent \textbf{Docente:} Carlos R P Tovar \\

\section*{Título}
\textbf{Análisis sistémico de una organización: de los paradigmas a la dinámica de sistemas.}

\section*{Objetivo}
Al finalizar la actividad, el estudiante demuestra su comprensión de los contenidos vistos en las semanas 1--5 aplicándolos al análisis de una organización real o ficticia, integrando paradigmas, organización de sistemas complejos, arquetipos sistémicos y dinámicos, y los fundamentos de dinámica de sistemas (descripción, situación problemática, levantamiento de información, elementos y relaciones).

\section*{Indicaciones y entregable}
\begin{enumerate}
  \item \textbf{Selecciona} una organización (empresa, unidad universitaria, institución pública, ONG o comunidad). Puede ser real o ficticia, pero debe describirse con claridad.
  \item \textbf{Parte I -- Paradigmas y TGS (Semana 1):}
  \begin{itemize}
    \item Explica cuál paradigma (reduccionista vs. sistémico) predomina en el análisis tradicional de la organización.
    \item Describe cómo se aplican los principios de la TGS: \emph{causalidad, teleología, recursividad y manejo de la información}.
  \end{itemize}
  \item \textbf{Parte II -- Organización de sistemas complejos (Semana 2):}
  \begin{itemize}
    \item Identifica: \textbf{supra-sistema}, al menos \textbf{dos infra-sistemas}, un \textbf{iso-sistema} y un \textbf{hetero-sistema}.
    \item Señala los \textbf{subsistemas} de la organización: \emph{social, técnico y organizacional}.
  \end{itemize}
  \item \textbf{Parte III -- Arquetipos sistémicos (Semana 3):}
  \begin{itemize}
    \item Elige \textbf{dos} arquetipos (p.ej., \emph{límites del crecimiento, erosión de metas, soluciones rápidas que fallan, éxito para quien tiene éxito, escalada, tragedia del terreno común}).
    \item Explica cómo podrían manifestarse en la organización y sus consecuencias.
  \end{itemize}
  \item \textbf{Parte IV -- Arquetipos dinámicos (Semana 4):}
  \begin{itemize}
    \item Clasifica un proceso clave como \textbf{estable}, \textbf{inestable} u \textbf{oscilante}.
    \item Describe un \textbf{retardo/demora} y su efecto en el comportamiento del sistema.
  \end{itemize}
  \item \textbf{Parte V -- Dinámica de sistemas (Semana 5):}
  \begin{itemize}
    \item Redacta la \textbf{situación problemática} más relevante.
    \item Identifica \textbf{al menos cinco elementos} y \textbf{tres relaciones causales} (puedes representarlas como bucles causales).
  \end{itemize}
  \item \textbf{Formato de entrega:} Informe en PDF de 5--8 páginas, con portada, índice, desarrollo y referencias. Incluye al menos un gráfico (mapa de sistemas/diagrama causal).
\end{enumerate}

\section*{Rúbrica de evaluación (20\% del curso)}
\begin{center}
\begin{tabular}{p{8cm}cc}
\toprule
\textbf{Criterio} & \textbf{Peso} & \textbf{Escala}\\
\midrule
Claridad y precisión conceptual (definiciones, principios TGS) & 20\% & 0--20 \\
Identificación de sistemas y subsistemas (supra/infra/iso/hetero; social/técnico/organizacional) & 20\% & 0--20 \\
Aplicación de arquetipos sistémicos y dinámicos (casos bien justificados) & 20\% & 0--20 \\
Situación problemática, elementos y relaciones causales (calidad y coherencia del modelo) & 20\% & 0--20 \\
Estructura, redacción académica y soportes gráficos / referencias & 20\% & 0--20 \\
\bottomrule
\end{tabular}
\end{center}

\section*{Recomendaciones}
\begin{itemize}
  \item Usa bibliografía del curso (Herrscher; Osorio Gómez; Garciandía Imaz) y/o fuentes adicionales pertinentes.
  \item Si representas bucles causales, identifica claramente variables, el sentido de las flechas y posibles retardos.
  \item Verifica coherencia entre diagnóstico, arquetipos identificados y propuesta de análisis.
\end{itemize}

\vfill
\noindent \textbf{Fecha de entrega:} \underline{\hspace{3cm}} \hfill \textbf{Modalidad:} Aula virtual / presencial\\

\end{document}
