\documentclass[12pt]{article}
\usepackage[spanish]{babel}
\usepackage[utf8]{inputenc}
\usepackage[T1]{fontenc}
\usepackage{listings}
\usepackage{xcolor}
\usepackage{geometry}
\usepackage{tcolorbox}
\usepackage{enumitem}

\geometry{a4paper, margin=2cm}

% Configuración de colores
\definecolor{codebg}{RGB}{240, 240, 240}
\definecolor{keyword}{RGB}{0, 90, 156}
\definecolor{comment}{RGB}{34, 139, 34}

% Configuración para Haskell
\lstdefinestyle{haskell}{
    language=Haskell,
    basicstyle=\ttfamily\small,
    keywordstyle=\color{keyword}\bfseries,
    commentstyle=\color{comment},
    backgroundcolor=\color{codebg},
    showstringspaces=false,
    frame=single,
    rulecolor=\color{black}
}

% Caja para ejercicios
\newtcolorbox{ejercicio}[1]{
    colback=white,
    colframe=blue!50!black,
    fonttitle=\bfseries,
    title=#1,
    boxsep=5pt,
    arc=4pt,
    breakable
}

\title{Guía de Laboratorio: Functores y Aplicaciones de Listas}
\author{Programación Lógica y Funcional \\ Universidad Tecnológica del Perú}
\date{Docente: Carlos R. P. Tovar}

\begin{document}

\maketitle

\section*{Objetivos de la Sesión}
\begin{itemize}
\item Comprender el concepto de functor en Haskell
\item Aplicar functores para transformar datos en contextos
\item Utilizar funciones de orden superior con listas
\item Implementar ejemplos prácticos de functores con listas
\item Desarrollar habilidades para procesamiento de datos con functores
\end{itemize}

\section*{Material Requerido}
\begin{itemize}
\item GHC (Glasgow Haskell Compiler) instalado
\item Editor de texto con soporte para Haskell
\item Esta guía de laboratorio
\end{itemize}

\begin{ejercicio}{Ejercicio 1: Introducción a Functores (20 min)}
Implementa las instancias de Functor para tipos personalizados y utiliza fmap.

\begin{lstlisting}[style=haskell]
-- Tipo Maybe ya tiene instancia de Functor
ejemploMaybe :: Maybe Int -> Maybe Int
ejemploMaybe = fmap (+1)

-- Tipo Either ya tiene instancia de Functor
ejemploEither :: Either String Int -> Either String Int
ejemploEither = fmap (*2)

-- Crea tu propio tipo y su instancia de Functor
data Resultado a = Exito a | Error String deriving Show

instance Functor Resultado where
    fmap f (Exito x) = Exito (f x)
    fmap _ (Error msg) = Error msg

-- Ejemplos de uso:
-- fmap (+1) (Exito 5) -> Exito 6
-- fmap (+1) (Error "problema") -> Error "problema"
\end{lstlisting}

\textbf{Tareas:}
\begin{enumerate}
\item Completa la instancia de Functor para Resultado
\item Crea una función que transforme una lista de Resultados
\item Implementa una función que aplique una transformación a un Resultado sólo si es Exito
\end{enumerate}
\end{ejercicio}

\begin{ejercicio}{Ejercicio 2: Aplicaciones de Listas con Map y Filter (25 min)}
Trabaja con funciones de orden superior para procesar listas.

\begin{lstlisting}[style=haskell]
-- Datos de ejemplo
nums = [1, 2, 3, 4, 5, 6, 7, 8, 9, 10]

noms = ["Ana", "Carlos", "Elena", "David", "Beatriz"]

-- 1. Transformaciones con map
cuads = map (^2) nums

inits = map head noms

-- 2. Filtrado con filter
pars = filter even nums

nomsLargos = filter (\x -> length x > 5) noms

-- 3. Combinaciones
cuadsPars = map (^2) (filter even nums)
\end{lstlisting}

\textbf{Tareas:}
\begin{enumerate}
\item Crea una función que devuelva los cubos de los números impares
\item Filtra los nombres que contienen la letra 'a' y conviértelos a mayúsculas
\item Implementa una función que sume 10 a los números mayores que 5
\end{enumerate}
\end{ejercicio}

\begin{ejercicio}{Ejercicio 3: Functores con Listas Anidadas (20 min)}
Trabaja con estructuras de datos anidadas y functores.

\begin{lstlisting}[style=haskell]
-- Listas anidadas
listAnid = [[1, 2, 3], [4, 5], [6, 7, 8, 9]]

-- Transformación con doble fmap
dupAnid = fmap (fmap (*2)) listAnid

-- Aplanar lista
aplana = concat

-- Functor para procesamiento en profundidad
procProfundo = fmap (fmap show)
\end{lstlisting}

\textbf{Tareas:}
\begin{enumerate}
\item Crea una función que calcule la suma de cada sublista
\item Implementa una función que filtre las sublistas con más de 2 elementos
\item Crea una transformación que convierta todos los números a string con prefijo "N-"
\end{enumerate}
\end{ejercicio}

\begin{ejercicio}{Ejercicio 4: Aplicaciones Prácticas con Datos Reales (25 min)}
Aplica functores y listas a un escenario del mundo real.

\begin{lstlisting}[style=haskell]
-- Tipo para representar productos
data Producto = Producto {
    nom :: String,
    pre :: Double,
    cant :: Int
} deriving Show

-- Datos de ejemplo
inv = [
    Producto "Laptop" 1200.50 5,
    Producto "Mouse" 25.99 20,
    Producto "Teclado" 75.30 12,
    Producto "Monitor" 350.00 8
]

-- Función auxiliar para valor
valorProd p = pre p * fromIntegral (cant p)

-- 1. Obtener todos los nombres
nomsProds = map nom inv

-- 2. Filtrar productos caros
prodsCaros = filter (\p -> pre p > 100.0) inv

-- 3. Calcular valor total por producto
vInventario = map valorProd inv

-- 4. Aplicar descuento del 10%
aplDesc = map (\p -> p {pre = pre p * 0.9})
\end{lstlisting}

\textbf{Tareas:}
\begin{enumerate}
\item Calcula el valor total del inventario
\item Filtra los productos con menos de 10 unidades
\item Crea una lista de strings con formato "Nombre: Precio"
\item Implementa un descuento escalonado
\end{enumerate}
\end{ejercicio}

\section*{Ejercicios Adicionales}
\begin{lstlisting}[style=haskell]
-- 1. Functor para tipos personalizados complejos
data Arbol a = Hoja a | Nodo (Arbol a) (Arbol a) deriving Show

instance Functor Arbol where
    fmap f (Hoja x) = Hoja (f x)
    fmap f (Nodo izq der) = Nodo (fmap f izq) (fmap f der)

-- 2. Procesamiento de texto
procTexto = fmap (filter (/= ' ') . map toUpper)

-- 3. Calculadora estadística
stats xs = (prom, min, max)
    where
        prom = sum xs / fromIntegral (length xs)
        min = minimum xs
        max = maximum xs
\end{lstlisting}

\section*{Reto Final}
\begin{lstlisting}[style=haskell]
-- Sistema de procesamiento de pedidos
data Pedido = Pedido {
    cli :: String,
    items :: [String],
    tot :: Double,
    desc :: Bool
} deriving Show

peds = [
    Pedido "Ana" ["Laptop", "Mouse"] 1226.49 True,
    Pedido "Carlos" ["Teclado"] 75.30 False,
    Pedido "Elena" ["Monitor", "Teclado"] 425.30 True
]

-- Tareas del reto:
-- 1. Aplicar 15% de descuento a pedidos con descuento
-- 2. Filtrar pedidos con total mayor a 500.00
-- 3. Crear resumen por cliente
-- 4. Calcular el total general de todos los pedidos
\end{lstlisting}

\section*{Evaluación}
\begin{itemize}
\item Correcta implementación de instancias de Functor
\item Uso apropiado de map, filter y funciones de orden superior
\item Manejo adecuado de listas anidadas
\item Soluciones eficientes y elegantes
\item Manejo de casos bordes y validaciones
\end{itemize}

\section*{Recursos Adicionales}
\begin{itemize}
\item Documentación oficial de Haskell: https://www.haskell.org/
\item Hoogle para búsqueda de funciones: https://hoogle.haskell.org/
\item Learn You a Haskell: http://learnyouahaskell.com/
\end{itemize}

\end{document}