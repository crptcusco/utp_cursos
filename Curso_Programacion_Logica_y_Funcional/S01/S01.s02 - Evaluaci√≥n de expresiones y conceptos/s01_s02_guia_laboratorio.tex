\documentclass[12pt]{article}
\usepackage[spanish]{babel}
\usepackage[utf8]{inputenc}
\usepackage[T1]{fontenc}
\usepackage{listings}
\usepackage{xcolor}
\usepackage{geometry}
\usepackage{hyperref}
\usepackage{tcolorbox}
\usepackage{amsmath}
\usepackage{graphicx}

% Configuración de página
\geometry{a4paper, margin=2cm}
\setlength{\parindent}{0pt}

% Colores para código
\definecolor{haskell}{RGB}{94, 80, 134}
\definecolor{prolog}{RGB}{0, 90, 156}
\definecolor{background}{RGB}{240, 245, 250}

% Configuración para Haskell
\lstdefinestyle{haskell}{
    language=Haskell,
    basicstyle=\ttfamily\small,
    keywordstyle=\color{haskell}\bfseries,
    commentstyle=\color{gray},
    numbers=left,
    showstringspaces=false,
    backgroundcolor=\color{background},
    frame=single,
    rulecolor=\color{haskell}
}

% Configuración para Prolog
\lstdefinestyle{prolog}{
    language=Prolog,
    basicstyle=\ttfamily\small,
    keywordstyle=\color{prolog}\bfseries,
    commentstyle=\color{gray},
    numbers=left,
    showstringspaces=false,
    backgroundcolor=\color{background},
    frame=single,
    rulecolor=\color{prolog}
}

% Caja para ejercicios
\newtcolorbox{ejercicio}[1]{
    colback=white,
    colframe=blue!50!black,
    fonttitle=\bfseries,
    title=#1,
    boxsep=5pt,
    arc=4pt
}

\title{Programación Lógica y Funcional: Sesión 2}
\author{Docente: Carlos R. P. Tovar}
\date{}

\begin{document}

\maketitle

\section*{Introducción}
En esta sesión practicaremos:
\begin{itemize}
\item Evaluación de expresiones en Haskell
\item Consultas simples y complejas en Prolog
\item Comparación entre paradigmas funcional y lógico
\end{itemize}

\section*{1. Ejercicio Haskell: Evaluación de Expresiones}
\begin{ejercicio}{Función Básica}
\textbf{Objetivo:} Comprender la reducción de expresiones en programación funcional.

\begin{lstlisting}[style=haskell]
doble x = x * 2
doble (3 + 2) - 1
\end{lstlisting}

\textbf{Pasos de solución:}
\begin{enumerate}
\item Resolver paréntesis: \(3 + 2 = 5\)
\item Aplicar función: \(\texttt{doble}\ 5 = 5 \times 2 = 10\)
\item Operación final: \(10 - 1 = 9\)
\end{enumerate}
\end{ejercicio}

\section*{2. Ejercicio Prolog: Consultas Simples}
\begin{ejercicio}{Relaciones Familiares}
\textbf{Objetivo:} Consultar relaciones en una base de conocimientos.

\begin{lstlisting}[style=prolog]
% Base de conocimientos
padre(juan, maria).
padre(maria, carlos).
abuelo(X, Y) :- padre(X, Z), padre(Z, Y).

% Consultas
?- padre(juan, maria).     % true
?- abuelo(juan, carlos).   % true
\end{lstlisting}

\textbf{Explicación:}
\begin{itemize}
\item Los \textbf{hechos} definen relaciones directas
\item Las \textbf{reglas} permiten inferir relaciones complejas
\end{itemize}
\end{ejercicio}

\section*{3. Ejercicio Integrado: Factorial}
\begin{ejercicio}{Comparación de Paradigmas}
\begin{minipage}{0.48\textwidth}
\textbf{Haskell (Funcional)}
\begin{lstlisting}[style=haskell]
factorial 0 = 1
factorial n = 
  n * factorial (n-1)
\end{lstlisting}
\end{minipage}
\hfill
\begin{minipage}{0.48\textwidth}
\textbf{Prolog (Lógico)}
\begin{lstlisting}[style=prolog]
factorial(0, 1).
factorial(N, F) :- 
  N > 0, 
  M is N-1, 
  factorial(M, FM), 
  F is N * FM.
\end{lstlisting}
\end{minipage}

\vspace{1em}
\textbf{Puntos clave:}
\begin{itemize}
\item Haskell: \textbf{Recursión} y \textbf{inmutabilidad}
\item Prolog: \textbf{Unificación} y \textbf{backtracking}
\end{itemize}
\end{ejercicio}

\section*{4. Tarea}
\begin{ejercicio}{Ejercicios Propuestos}
\textbf{1. Implementar en Haskell:}
\begin{lstlisting}[style=haskell]
sumaCuadrados :: [Int] -> Int
-- Calcula la suma de cuadrados de una lista
$[1,2,3] \rightarrow 1^{2} + 2^{2} + 3^{2} = 14$
\end{lstlisting}

\textbf{2. Modelar en Prolog:}
\begin{lstlisting}[style=prolog]
% Regla "hermano" usando padre y madre
% Dos personas son hermanas si comparten ambos padres
hermano(X, Y) :- ...
\end{lstlisting}
\end{ejercicio}

\section*{Recursos Adicionales}
\begin{itemize}
\item \href{https://swish.swi-prolog.org/}{SWISH Prolog Online}
\item \href{https://play.haskell.org/}{Haskell Playground}
\item Código completo: \href{https://github.com/}{GitHub Classroom Link}
\end{itemize}

\newpage

\section*{Apéndice: Soluciones}
\begin{ejercicio}{Solución Haskell (Suma de Cuadrados)}
\begin{lstlisting}[style=haskell]
sumaCuadrados :: [Int] -> Int
sumaCuadrados [] = 0
sumaCuadrados (x:xs) = (x * x) + sumaCuadrados xs
\end{lstlisting}
\end{ejercicio}

\begin{ejercicio}{Solución Prolog (Hermanos)}
\begin{lstlisting}[style=prolog]
hermano(X, Y) :- 
    padre(Z, X), 
    padre(Z, Y), 
    madre(W, X), 
    madre(W, Y), 
    X \= Y.
\end{lstlisting}
\end{ejercicio}

\begin{center}
\Large ¡Éxito en su práctica de laboratorio! \texttt{:-)}
\end{center}

\end{document}