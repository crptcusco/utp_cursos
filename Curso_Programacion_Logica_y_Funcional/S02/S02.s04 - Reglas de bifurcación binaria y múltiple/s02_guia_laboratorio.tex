\documentclass[12pt]{article}
\usepackage[utf8]{inputenc}
\usepackage{geometry}
\geometry{a4paper, margin=1in}
\usepackage{listings}
\usepackage{xcolor}

\lstset{
    basicstyle=\ttfamily\small,
    keywordstyle=\color{blue}\bfseries,
    commentstyle=\color{green!50!black},
    stringstyle=\color{red},
    showstringspaces=false,
    frame=single,
    breaklines=true
}

\title{Guía de Laboratorio - Programación Declarativa}
\author{Curso de Programación Lógica y Funcional}
\date{}

\begin{document}

\maketitle

\section*{Ejercicios en Haskell}

\subsection*{1. Determinar si un número es par}
\begin{lstlisting}[language=Haskell]
esPar :: Int -> Bool
esPar n = n `mod` 2 == 0

-- Ejemplo:
-- esPar 4  => True
-- esPar 5  => False
\end{lstlisting}

\subsection*{2. Calcular el cuadrado de un número}
\begin{lstlisting}[language=Haskell]
cuadrado :: Int -> Int
cuadrado x = x * x

-- Ejemplo:
-- cuadrado 5 => 25
\end{lstlisting}

\subsection*{3. Obtener el factorial de un número}
\begin{lstlisting}[language=Haskell]
factorial :: Int -> Int
factorial 0 = 0
factorial 1 = 1
factorial n = n * factorial (n - 1)

-- Ejemplo:
-- factorial 5 => 120
\end{lstlisting}

\subsection*{4. Sumar todos los elementos de una lista}
\begin{lstlisting}[language=Haskell]
sumaLista :: [Int] -> Int
sumaLista [] = 0
sumaLista (x:xs) = x + sumaLista xs

-- Ejemplo:
-- sumaLista [1,2,3,4] => 10
\end{lstlisting}

\subsection*{5. Obtener el máximo de una lista}
\begin{lstlisting}[language=Haskell]
maximo :: [Int] -> Int
maximo [x] = x
maximo (x:xs) = max x (maximo xs)

-- Ejemplo:
-- maximo [3,8,2,5] => 8
\end{lstlisting}

\newpage
\section*{Ejercicios en Prolog}

\subsection*{1. Definir hechos de animales y clasificarlos como mamíferos o aves}
\begin{lstlisting}[language=Prolog]
mamifero(perro).
mamifero(gato).
ave(gallina).
ave(paloma).

es_mamifero(X) :- mamifero(X).
es_ave(X) :- ave(X).

% Consultas:
% ?- es_mamifero(perro).
% ?- es_ave(paloma).
\end{lstlisting}

\subsection*{2. Regla que determine si un número es par o impar}
\begin{lstlisting}[language=Prolog]
par(N) :- 0 is N mod 2.
impar(N) :- 1 is N mod 2.

% Consultas:
% ?- par(4).
% ?- impar(7).
\end{lstlisting}

\subsection*{3. Reglas para identificar si alguien es hijo de otra persona}
\begin{lstlisting}[language=Prolog]
padre(juan, maria).
madre(ana, maria).

hijo(H, P) :- padre(P, H).
hijo(H, M) :- madre(M, H).

% Consultas:
% ?- hijo(maria, juan).
% ?- hijo(maria, ana).
\end{lstlisting}

\subsection*{4. Modelar una familia pequeña con relaciones de padre, madre e hijo}
\begin{lstlisting}[language=Prolog]
padre(carlos, pedro).
madre(luisa, pedro).
padre(carlos, ana).
madre(luisa, ana).

hijo(H, P) :- padre(P, H).
hijo(H, M) :- madre(M, H).

% Consultas:
% ?- hijo(pedro, carlos).
% ?- hijo(ana, luisa).
\end{lstlisting}

\subsection*{5. Representar una lista de números y consultar el primero y el último elemento}
\begin{lstlisting}[language=Prolog]
lista([1,2,3,4,5]).

primero([H|_], H).
ultimo([X], X).
ultimo([_|T], X) :- ultimo(T, X).

% Consultas:
% ?- lista(L), primero(L, P).
% ?- lista(L), ultimo(L, U).
\end{lstlisting}

\end{document}
