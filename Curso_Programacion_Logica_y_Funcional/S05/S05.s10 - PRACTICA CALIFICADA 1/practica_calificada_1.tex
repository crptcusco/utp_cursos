\documentclass[12pt]{article}
\usepackage[spanish]{babel}
\usepackage[utf8]{inputenc}
\usepackage[T1]{fontenc}
\usepackage{geometry}
\usepackage{listings}
\usepackage{xcolor}

\geometry{a4paper, margin=2.5cm}

% Configuración de código
\definecolor{codebg}{RGB}{245,245,245}
\definecolor{keyword}{RGB}{0,0,180}
\definecolor{comment}{RGB}{0,120,0}
\lstset{
  basicstyle=\ttfamily\small,
  backgroundcolor=\color{codebg},
  keywordstyle=\color{keyword}\bfseries,
  commentstyle=\color{comment},
  numbers=left,
  frame=single,
  showstringspaces=false
}

\title{\textbf{Práctica Calificada 1} \\ Programación Lógica y Funcional}
\author{Docente: Carlos R. P. Tovar}
\date{Ciclo 2025-2}

\begin{document}
\maketitle

\section*{Indicaciones}
\begin{itemize}
    \item Duración: 90 minutos.
    \item Responda justificando cada solución con código en Haskell.
    \item Puede usar \texttt{ghci} para validar sus funciones.
    \item Cada pregunta vale 7 puntos (total: 21).
\end{itemize}

\section*{Ejercicio 1 (7 puntos)}
Implemente en Haskell una función:

\begin{lstlisting}[language=Haskell]
clasificarNota :: Int -> String
\end{lstlisting}

que clasifique una nota de 0 a 20:
\begin{itemize}
    \item 0 -- 10 : "Desaprobado"
    \item 11 -- 13 : "Regular"
    \item 14 -- 17 : "Bueno"
    \item 18 -- 20 : "Excelente"
    \item Fuera de rango : "Nota inválida"
\end{itemize}

\textbf{Prueba esperada:}
\begin{lstlisting}[language=Haskell]
ghci> clasificarNota 9
"Desaprobado"
ghci> clasificarNota 15
"Bueno"
ghci> clasificarNota 21
"Nota invalida"
\end{lstlisting}

\section*{Ejercicio 2 (7 puntos)}
Defina en Haskell una función recursiva:

\begin{lstlisting}[language=Haskell]
contarDigitos :: Int -> Int
\end{lstlisting}

que retorne la cantidad de dígitos de un número entero positivo.

\textbf{Ejemplo de uso:}
\begin{lstlisting}[language=Haskell]
ghci> contarDigitos 7
1
ghci> contarDigitos 12345
5
\end{lstlisting}

\textbf{Extensión:} si el número es negativo, conviértalo primero a positivo.

\section*{Ejercicio 3 (7 puntos)}
Implemente un programa principal recursivo:

\begin{lstlisting}[language=Haskell]
menu :: IO ()
\end{lstlisting}

que muestre las siguientes opciones:

\begin{enumerate}
    \item Calcular factorial de un número
    \item Calcular el cuadrado de un número
    \item Salir
\end{enumerate}

El menú debe repetirse hasta que el usuario elija la opción 3 (Salir).

\textbf{Ejemplo de ejecución:}
\begin{verbatim}
Seleccione una opción:
1. Factorial
2. Cuadrado
3. Salir
> 1
Ingrese un número: 5
El factorial es 120

Seleccione una opción:
1. Factorial
2. Cuadrado
3. Salir
> 3
Adiós!
\end{verbatim}

\section*{Criterios de Evaluación}
\begin{itemize}
    \item Uso correcto de funciones y guards (Ej. 1).
    \item Implementación recursiva sin usar funciones predefinidas (Ej. 2).
    \item Manejo de recursion y E/S en menú interactivo (Ej. 3).
\end{itemize}

\end{document}
