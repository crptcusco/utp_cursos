\documentclass[12pt,a4paper]{article}
\usepackage[spanish]{babel}
\usepackage[utf8]{inputenc}
\usepackage[T1]{fontenc}
\usepackage{enumitem}
\usepackage{amsmath}
\usepackage{listings}
\usepackage{xcolor}
\usepackage{geometry}
\geometry{margin=2.5cm}

\lstset{
    language=Python,
    basicstyle=\ttfamily\small,
    keywordstyle=\color{blue},
    stringstyle=\color{red},
    commentstyle=\color{gray},
    showstringspaces=false,
    numbers=left,
    numberstyle=\tiny,
    stepnumber=1,
    inputencoding=utf8,
    extendedchars=true,
    frame=single,
    literate={á}{{\'a}}1
           {é}{{\'e}}1
           {í}{{\'i}}1
           {ó}{{\'o}}1
           {ú}{{\'u}}1
           {ñ}{{\~n}}1
           {Á}{{\'A}}1
           {É}{{\'E}}1
           {Í}{{\'I}}1
           {Ó}{{\'O}}1
           {Ú}{{\'U}}1
           {Ñ}{{\~N}}1
}

\title{\textbf{Guía de Laboratorio – Sesión 06} \\ Estructuras de Datos en Python}
\author{Curso de Inteligencia Artificial}
\date{}

\begin{document}
\maketitle

\section*{Objetivo de la sesión}
Al finalizar la sesión, el alumno será capaz de:
\begin{itemize}
    \item Crear y manipular listas, tuplas, conjuntos y diccionarios en Python.
    \item Acceder, modificar y eliminar elementos de estructuras de datos.
    \item Usar comprensiones de listas para generar colecciones.
    \item Aplicar operaciones básicas y métodos propios de cada estructura.
\end{itemize}

\section*{Instrucciones}
Resuelva los siguientes ejercicios en Python. Guarde cada uno en un archivo independiente.

\section*{Ejercicios}
\begin{enumerate}[label=\textbf{Ejercicio \arabic*:}, leftmargin=1.5cm]
    \item \textbf{Lista de compras:} Cree una lista con al menos 5 productos. Agregue uno más, elimine uno y muestre la lista final.
    \item \textbf{Suma de elementos:} Dada una lista de números, calcule la suma usando un ciclo.
    \item \textbf{Tupla de coordenadas:} Cree una tupla con coordenadas (x, y) e imprímala con formato.
    \item \textbf{Intercambio de valores:} Use una tupla para intercambiar los valores de dos variables.
    \item \textbf{Conjunto de estudiantes:} Cree un conjunto con nombres de estudiantes sin duplicados y agregue un nuevo nombre.
    \item \textbf{Unión e intersección de conjuntos:} Dados dos conjuntos de enteros, calcule su unión e intersección.
    \item \textbf{Diccionario de calificaciones:} Cree un diccionario con nombres de alumnos y notas, y agregue un nuevo alumno.
    \item \textbf{Búsqueda en diccionario:} Dado un nombre, busque y muestre su calificación.
    \item \textbf{Conteo de letras:} Dada una cadena, use un diccionario para contar cuántas veces aparece cada letra.
    \item \textbf{Comprensión de listas:} Genere una lista con los cuadrados de los números del 1 al 10 usando list comprehension.
\end{enumerate}

\section*{Recomendaciones}
\begin{itemize}
    \item Use comentarios para explicar cada parte de su código.
    \item Pruebe con diferentes datos de entrada.
    \item Explore métodos como \texttt{append()}, \texttt{remove()}, \texttt{keys()}, \texttt{values()}, \texttt{union()}, \texttt{intersection()}.
\end{itemize}

\section*{Ejercicios con soluciones}
\begin{enumerate}[label=\textbf{Ejercicio \arabic*:}, leftmargin=1.5cm]
    \item \textbf{Lista de compras:}
    \begin{lstlisting}
compras = ["pan", "leche", "huevos", "arroz", "pollo"]
compras.append("queso")
compras.remove("arroz")
print(compras)
    \end{lstlisting}

    \item \textbf{Suma de elementos:}
    \begin{lstlisting}
numeros = [1, 2, 3, 4, 5]
suma = 0
for n in numeros:
    suma += n
print(suma)
    \end{lstlisting}

    \item \textbf{Tupla de coordenadas:}
    \begin{lstlisting}
coordenadas = (10, 20)
print(f"Coordenadas: x={coordenadas[0]}, y={coordenadas[1]}")
    \end{lstlisting}

    \item \textbf{Intercambio de valores:}
    \begin{lstlisting}
a, b = 5, 10
a, b = b, a
print(a, b)
    \end{lstlisting}

    \item \textbf{Conjunto de estudiantes:}
    \begin{lstlisting}
estudiantes = {"Ana", "Luis", "Pedro"}
estudiantes.add("Marta")
print(estudiantes)
    \end{lstlisting}

    \item \textbf{Unión e intersección de conjuntos:}
    \begin{lstlisting}
A = {1, 2, 3}
B = {3, 4, 5}
print(A.union(B))
print(A.intersection(B))
    \end{lstlisting}

    \item \textbf{Diccionario de calificaciones:}
    \begin{lstlisting}
notas = {"Ana": 15, "Luis": 18, "Pedro": 12}
notas["Marta"] = 17
print(notas)
    \end{lstlisting}

    \item \textbf{Búsqueda en diccionario:}
    \begin{lstlisting}
nombre = "Luis"
if nombre in notas:
    print(notas[nombre])
else:
    print("Alumno no encontrado")
    \end{lstlisting}

    \item \textbf{Conteo de letras:}
    \begin{lstlisting}
texto = "python"
conteo = {}
for letra in texto:
    conteo[letra] = conteo.get(letra, 0) + 1
print(conteo)
    \end{lstlisting}

    \item \textbf{Comprensión de listas:}
    \begin{lstlisting}
cuadrados = [x**2 for x in range(1, 11)]
print(cuadrados)
    \end{lstlisting}
\end{enumerate}

\end{document}
