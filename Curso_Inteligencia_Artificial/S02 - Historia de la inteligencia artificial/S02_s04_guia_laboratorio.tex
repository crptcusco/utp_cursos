\documentclass[12pt,a4paper]{article}
\usepackage[spanish]{babel}
\usepackage[utf8]{inputenc}
\usepackage[T1]{fontenc}
\usepackage{enumitem}
\usepackage{amsmath}
\usepackage{listings}
\usepackage{xcolor}
\usepackage{geometry}
\geometry{margin=2.5cm}

\lstset{
    language=Python,
    basicstyle=\ttfamily\small,
    keywordstyle=\color{blue},
    stringstyle=\color{red},
    commentstyle=\color{gray},
    showstringspaces=false,
    numbers=left,
    numberstyle=\tiny,
    stepnumber=1,
    frame=single
}

\title{\textbf{Guía de Laboratorio – Sesión 04} \\ Estructuras de Control en Python}
\author{Curso de Inteligencia Artificial}
\date{}

\begin{document}
\maketitle

\section*{Objetivo de la sesión}
Al finalizar la sesión, el alumno será capaz de:
\begin{itemize}
    \item Utilizar condicionales \texttt{if}, \texttt{elif} y \texttt{else}.
    \item Implementar bucles \texttt{for} y \texttt{while}.
    \item Emplear operadores lógicos y relacionales.
    \item Aplicar control de flujo para resolver problemas prácticos.
\end{itemize}

\section*{Instrucciones}
Resuelva los siguientes ejercicios en Python. Trabaje cada uno en un archivo independiente para facilitar su prueba.

\section*{Ejercicios}

\begin{enumerate}[label=\textbf{Ejercicio \arabic*:}, leftmargin=1.5cm]
    \item \textbf{Número positivo, negativo o cero:} Lea un número y determine si es positivo, negativo o cero.
    
    \item \textbf{Mayor de tres números:} Lea tres números y muestre cuál es el mayor.
    
    \item \textbf{Par o impar:} Lea un número entero y determine si es par o impar.
    
    \item \textbf{Año bisiesto:} Lea un año y determine si es bisiesto.
    
    \item \textbf{Tabla de multiplicar:} Lea un número e imprima su tabla de multiplicar del 1 al 10 usando un \texttt{for}.
    
    \item \textbf{Contador descendente:} Imprima los números del 10 al 1 usando un \texttt{while}.
    
    \item \textbf{Suma de números:} Solicite números al usuario hasta que ingrese 0, luego muestre la suma total.
    
    \item \textbf{Adivinar número:} Genere un número aleatorio del 1 al 10 y permita al usuario adivinarlo hasta acertar. Muestre mensajes de ayuda.
    
    \item \textbf{Calculadora de factorial:} Lea un número entero y calcule su factorial con un bucle.
    
    \item \textbf{Menú interactivo:} Implemente un menú que permita:
    \begin{itemize}
        \item Sumar dos números.
        \item Restar dos números.
        \item Salir del programa.
    \end{itemize}
    Utilice un \texttt{while} para mantener el menú activo.
\end{enumerate}

\section*{Recomendaciones}
\begin{itemize}
    \item Utilice comentarios para explicar cada paso del código.
    \item Pruebe con distintos valores de entrada.
    \item No olvide validar las entradas para evitar errores.
\end{itemize}

\end{document}
