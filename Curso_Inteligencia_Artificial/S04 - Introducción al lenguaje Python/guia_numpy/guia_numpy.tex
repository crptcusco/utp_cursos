\documentclass[a4paper,12pt]{article}
\usepackage[utf8]{inputenc}
\usepackage[T1]{fontenc}
\usepackage{lmodern}
\usepackage[spanish]{babel}
\usepackage{amsmath}
\usepackage{amsfonts}
\usepackage{geometry}
\usepackage{listings}
\usepackage{xcolor}
\usepackage{hyperref}

\geometry{margin=1in}
\hypersetup{
    colorlinks=true,
    linkcolor=blue,
    filecolor=magenta,      
    urlcolor=cyan,
}

\definecolor{codegreen}{rgb}{0,0.6,0}
\definecolor{codegray}{rgb}{0.5,0.5,0.5}
\definecolor{codepurple}{rgb}{0.58,0,0.82}
\definecolor{backcolour}{rgb}{0.95,0.95,0.92}

\lstdefinestyle{mystyle}{
    backgroundcolor=\color{backcolour},   
    commentstyle=\color{codegreen},
    keywordstyle=\color{magenta},
    numberstyle=\tiny\color{codegray},
    stringstyle=\color{codepurple},
    basicstyle=\ttfamily\footnotesize,
    breakatwhitespace=false,         
    breaklines=true,                 
    captionpos=b,                    
    keepspaces=true,                 
    numbers=left,                    
    numbersep=5pt,                  
    showspaces=false,                
    showstringspaces=false,
    showtabs=false,                  
    tabsize=2,
    frame=single
}

\lstset{style=mystyle}

\title{Guía de Laboratorio: Introducción a NumPy}
\author{Docente: Carlos R. P. Tovar \\ Curso: Inteligencia Artificial}
\date{}

\begin{document}

\maketitle

\section{Introduccion}
NumPy es una biblioteca fundamental en Python para el manejo de arreglos numericos y operaciones matematicas eficientes. Es ampliamente utilizada en ciencia de datos, machine learning y computacion cientifica debido a su capacidad para realizar calculos vectorizados de manera rapida y eficiente. Esta guía de laboratorio esta disenada para introducir los conceptos basicos de NumPy, incluyendo la creacion de arreglos, operaciones matematicas y manipulacion de datos.

\section{Objetivos}
\begin{itemize}
    \item Comprender el concepto de arreglos en NumPy y su diferencia con listas de Python.
    \item Aprender a crear y manipular arreglos de diferentes dimensiones.
    \item Realizar operaciones matematicas y estadisticas basicas con NumPy.
    \item Explorar funciones de indexacion y segmentacion de arreglos.
    \item Aplicar NumPy en problemas basicos de inteligencia artificial.
\end{itemize}

\section{Requisitos previos}
\begin{itemize}
    \item Tener instalado Python (version 3.6 o superior).
    \item Instalar la biblioteca NumPy (\texttt{pip install numpy}).
    \item Un entorno de desarrollo como Jupyter Notebook, Visual Studio Code o cualquier editor de texto.
\end{itemize}

\section{Ejercicios Practicos}

\subsection{Ejercicio 1: Creacion de arreglos}
En este ejercicio, crearas diferentes tipos de arreglos utilizando NumPy.

\begin{enumerate}
    \item Crea un arreglo 1D con los numeros del 1 al 5.
    \item Crea un arreglo 2D de 2x3 con valores enteros consecutivos.
    \item Crea un arreglo de ceros de tamano 3x4.
    \item Crea un arreglo con 10 valores espaciados uniformemente entre 0 y 1.
\end{enumerate}

\textbf{Codigo de ejemplo:}

\begin{lstlisting}[language=Python]
import numpy as np

# Arreglo 1D
arr1 = np.array([1, 2, 3, 4, 5])
print("Arreglo 1D:", arr1)

# Arreglo 2D
arr2 = np.array([[1, 2, 3], [4, 5, 6]])
print("Arreglo 2D:\n", arr2)

# Arreglo de ceros
arr3 = np.zeros((3, 4))
print("Arreglo de ceros:\n", arr3)

# Arreglo con valores espaciados
arr4 = np.linspace(0, 1, 10)
print("Arreglo espaciado:\n", arr4)
\end{lstlisting}

\textbf{Tarea:} Modifica el codigo para crear un arreglo 2D de 4x2 con valores aleatorios entre 0 y 10.

\subsection{Ejercicio 2: Operaciones con arreglos}
Realiza operaciones matematicas basicas con arreglos de NumPy.

\begin{enumerate}
    \item Crea dos arreglos 1D de tamano 4 con valores enteros.
    \item Realiza la suma, resta, multiplicacion y division elemento por elemento.
    \item Calcula el producto punto de los dos arreglos.
\end{enumerate}

\textbf{Codigo de ejemplo:}

\begin{lstlisting}[language=Python]
import numpy as np

# Crear dos arreglos
a = np.array([1, 2, 3, 4])
b = np.array([5, 6, 7, 8])

# Operaciones elemento por elemento
suma = a + b
resta = a - b
multiplicacion = a * b
division = a / b

print("Suma:", suma)
print("Resta:", resta)
print("Multiplicacion:", multiplicacion)
print("Division:", division)

# Producto punto
producto_punto = np.dot(a, b)
print("Producto punto:", producto_punto)
\end{lstlisting}

\textbf{Tarea:} Calcula el cuadrado de cada elemento del arreglo \texttt{a} y la raiz cuadrada del arreglo \texttt{b}.

\subsection{Ejercicio 3: Indexacion y segmentacion}
Explora como acceder a elementos y subarreglos en NumPy.

\begin{enumerate}
    \item Crea un arreglo 2D de 3x3 con valores del 1 al 9.
    \item Accede al elemento en la posicion (1, 2).
    \item Extrae la primera fila y la ultima columna.
    \item Selecciona un subarreglo de 2x2 desde la esquina superior izquierda.
\end{enumerate}

\textbf{Codigo de ejemplo:}

\begin{lstlisting}[language=Python]
import numpy as np

# Crear arreglo 2D
arr = np.array([[1, 2, 3], [4, 5, 6], [7, 8, 9]])
print("Arreglo 2D:\n", arr)

# Acceder a un elemento
elemento = arr[1, 2]
print("Elemento en (1, 2):", elemento)

# Extraer primera fila
primera_fila = arr[0, :]
print("Primera fila:", primera_fila)

# Extraer ultima columna
ultima_columna = arr[:, -1]
print("Ultima columna:", ultima_columna)

# Subarreglo 2x2
subarreglo = arr[0:2, 0:2]
print("Subarreglo 2x2:\n", subarreglo)
\end{lstlisting}

\textbf{Tarea:} Extrae un subarreglo que contenga las filas 1 y 2, y las columnas 0 y 1.

\subsection{Ejercicio 4: Funciones estadisticas}
Usa funciones de NumPy para calcular estadisticas basicas.

\begin{enumerate}
    \item Crea un arreglo 1D con 10 valores aleatorios entre 1 y 100.
    \item Calcula la media, mediana, desviacion estandar y suma de los valores.
    \item Encuentra el valor maximo y su indice.
\end{enumerate}

\textbf{Codigo de ejemplo:}

\begin{lstlisting}[language=Python]
import numpy as np

# Crear arreglo con valores aleatorios
arr = np.random.randint(1, 101, 10)
print("Arreglo aleatorio:", arr)

# Calcular estadisticas
media = np.mean(arr)
mediana = np.median(arr)
desviacion = np.std(arr)
suma = np.sum(arr)
maximo = np.max(arr)
indice_max = np.argmax(arr)

print("Media:", media)
print("Mediana:", mediana)
print("Desviacion estandar:", desviacion)
print("Suma:", suma)
print("Valor maximo:", maximo, "en indice:", indice_max)
\end{lstlisting}

\textbf{Tarea:} Crea un arreglo 2D de 3x3 con valores aleatorios y calcula la suma de cada fila y columna.

\subsection{Ejercicio 5: Aplicacion en IA - Normalizacion de datos}
La normalizacion de datos es un paso crucial en el preprocesamiento para algoritmos de IA.

\begin{enumerate}
    \item Crea un arreglo 2D que simule un dataset de caracteristicas (100 muestras, 5 caracteristicas).
    \item Normaliza los datos para que cada caracteristica tenga media 0 y desviacion estandar 1.
    \item Verifica que la normalizacion fue correcta.
\end{enumerate}

\textbf{Codigo de ejemplo:}

\begin{lstlisting}[language=Python]
import numpy as np

# Crear dataset simulado (100 muestras, 5 caracteristicas)
np.random.seed(42)  # Para resultados reproducibles
X = np.random.randn(100, 5) * 10 + 5  # Media ~5, desviacion ~10
print("Dataset original - Media:", np.mean(X, axis=0), "Desviacion:", np.std(X, axis=0))

# Normalizacion (Standard Scaling)
X_normalized = (X - np.mean(X, axis=0)) / np.std(X, axis=0)

print("Dataset normalizado - Media:", np.mean(X_normalized, axis=0), "Desviacion:", np.std(X_normalized, axis=0))
\end{lstlisting}

\textbf{Tarea:} Modifica el codigo para normalizar los datos en el rango [0, 1] (Min-Max scaling).

\section{Conclusion}
En esta guia, has aprendido a crear y manipular arreglos en NumPy, realizar operaciones matematicas, indexar subarreglos y calcular estadisticas basicas. NumPy es una herramienta poderosa que optimiza el manejo de datos numericos en Python. Te recomendamos explorar mas funciones avanzadas como \texttt{np.where}, \texttt{np.concatenate} y operaciones con matrices para profundizar tus conocimientos.

\section{Recursos adicionales}
\begin{itemize}
    \item Documentacion oficial de NumPy: \url{https://numpy.org/doc/stable/}
    \item Tutoriales interactivos en Jupyter Notebook: \url{https://jupyter.org/}
    \item Curso gratuito de NumPy en DataCamp: \url{https://www.datacamp.com/courses/intro-to-python-for-data-science}
\end{itemize}

\end{document}