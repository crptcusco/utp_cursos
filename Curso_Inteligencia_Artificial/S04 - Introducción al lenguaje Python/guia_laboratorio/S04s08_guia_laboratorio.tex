\documentclass[12pt,a4paper]{article}
\usepackage[spanish]{babel}
\usepackage[utf8]{inputenc}
\usepackage[T1]{fontenc}
\usepackage{enumitem}
\usepackage{amsmath}
\usepackage{listings}
\usepackage{xcolor}
\usepackage{geometry}
\geometry{margin=2.5cm}

\lstset{
    language=Python,
    basicstyle=\ttfamily\small,
    keywordstyle=\color{blue},
    stringstyle=\color{red},
    commentstyle=\color{gray},
    showstringspaces=false,
    numbers=left,
    numberstyle=\tiny,
    stepnumber=1,
    inputencoding=utf8,
    extendedchars=true,
    frame=single,
    literate={á}{{\'a}}1
           {é}{{\'e}}1
           {í}{{\'i}}1
           {ó}{{\'o}}1
           {ú}{{\'u}}1
           {ñ}{{\~n}}1
           {Á}{{\'A}}1
           {É}{{\'E}}1
           {Í}{{\'I}}1
           {Ó}{{\'O}}1
           {Ú}{{\'U}}1
           {Ñ}{{\~N}}1
}

\title{\textbf{Guía de Laboratorio – Sesión 08} \\ Funciones y Modularidad en Python}
\author{Curso de Inteligencia Artificial}
\date{}

\begin{document}
\maketitle

\section*{Objetivo de la sesión}
Al finalizar la sesión, el alumno será capaz de:
\begin{itemize}
    \item Definir y utilizar funciones en Python.
    \item Implementar parámetros y valores de retorno.
    \item Organizar el código en módulos reutilizables.
    \item Aplicar buenas prácticas en la creación de funciones.
\end{itemize}

\section*{Instrucciones}
Resuelva los siguientes ejercicios en Python. Guarde cada uno en un archivo independiente.  
Cuando un ejercicio indique modularidad, cree un módulo \texttt{utilidades.py} y utilícelo en su script principal.

\section*{Ejercicios}

\begin{enumerate}[label=\textbf{Ejercicio \arabic*:}, leftmargin=1.5cm]
    \item \textbf{Saludo personalizado:} Cree una función que reciba un nombre y muestre un saludo.
    
    \item \textbf{Área de un círculo:} Defina una función que reciba el radio y devuelva el área. Use el módulo \texttt{math}.
    
    \item \textbf{Número par o impar:} Cree una función que determine si un número es par o impar.
    
    \item \textbf{Conversión de temperatura:} Escriba funciones para convertir de Celsius a Fahrenheit y viceversa.
    
    \item \textbf{Factorial con función:} Implemente el cálculo del factorial usando una función.
    
    \item \textbf{Módulo de utilidades matemáticas:} Cree un módulo \texttt{utilidades.py} con funciones para:
    \begin{itemize}
        \item Calcular el cuadrado de un número.
        \item Calcular el cubo de un número.
        \item Calcular la raíz cuadrada.
    \end{itemize}
    Importe y utilice el módulo en un script principal.
    
    \item \textbf{Calculadora modular:} Cree un módulo \texttt{calculadora.py} con funciones de suma, resta, multiplicación y división. Implemente un menú en un script principal para usarlas.
    
    \item \textbf{Contar vocales:} Defina una función que cuente el número de vocales en una cadena.
    
    \item \textbf{Máximo común divisor y mínimo común múltiplo:} Implemente ambas funciones y pruébelas con varios pares de números.
    
    \item \textbf{Agenda de contactos modular:} Cree un módulo con funciones para:
    \begin{itemize}
        \item Agregar un contacto (nombre y teléfono).
        \item Buscar un contacto por nombre.
        \item Mostrar todos los contactos.
    \end{itemize}
    Use un diccionario como estructura de almacenamiento.
\end{enumerate}

\section*{Recomendaciones}
\begin{itemize}
    \item Pruebe sus funciones con distintos valores de entrada.
    \item Comente el código para documentar el propósito de cada función.
    \item Mantenga el código limpio y organizado en módulos reutilizables.
\end{itemize}

\section*{Ejercicios con soluciones}

\begin{enumerate}[label=\textbf{Ejercicio \arabic*:}, leftmargin=1.5cm]
    \item \textbf{Saludo personalizado:}
    \begin{lstlisting}
def saludar(nombre):
    print(f"Hola, {nombre}!")

saludar("Carlos")
    \end{lstlisting}
    
    \item \textbf{Área de un círculo:}
    \begin{lstlisting}
import math

def area_circulo(radio):
    return math.pi * radio**2

print(area_circulo(5))
    \end{lstlisting}
    
    \item \textbf{Número par o impar:}
    \begin{lstlisting}
def es_par(num):
    return num % 2 == 0

print(es_par(4))  # True
print(es_par(7))  # False
    \end{lstlisting}
    
    \item \textbf{Conversión de temperatura:}
    \begin{lstlisting}
def celsius_a_fahrenheit(c):
    return c * 9/5 + 32

def fahrenheit_a_celsius(f):
    return (f - 32) * 5/9

print(celsius_a_fahrenheit(0))
print(fahrenheit_a_celsius(32))
    \end{lstlisting}
    
    \item \textbf{Factorial con función:}
    \begin{lstlisting}
def factorial(n):
    resultado = 1
    for i in range(1, n+1):
        resultado *= i
    return resultado

print(factorial(5))
    \end{lstlisting}
    
    \item \textbf{Módulo de utilidades matemáticas:}
    \texttt{utilidades.py}:
    \begin{lstlisting}
import math

def cuadrado(n):
    return n ** 2

def cubo(n):
    return n ** 3

def raiz_cuadrada(n):
    return math.sqrt(n)
    \end{lstlisting}
    
    Script principal:
    \begin{lstlisting}
import utilidades

print(utilidades.cuadrado(4))
print(utilidades.cubo(3))
print(utilidades.raiz_cuadrada(16))
    \end{lstlisting}
    
    \item \textbf{Calculadora modular:}
    \texttt{calculadora.py}:
    \begin{lstlisting}
def sumar(a,b): return a+b
def restar(a,b): return a-b
def multiplicar(a,b): return a*b
def dividir(a,b): return a/b if b != 0 else "Error: división por cero"
    \end{lstlisting}
    
    Script principal:
    \begin{lstlisting}
import calculadora

print(calculadora.sumar(3,4))
print(calculadora.dividir(10,2))
    \end{lstlisting}
    
    \item \textbf{Contar vocales:}
    \begin{lstlisting}
def contar_vocales(texto):
    vocales = "aeiouAEIOU"
    return sum(1 for letra in texto if letra in vocales)

print(contar_vocales("Inteligencia Artificial"))
    \end{lstlisting}
    
    \item \textbf{MCD y MCM:}
    \begin{lstlisting}
import math

def mcd(a,b):
    return math.gcd(a,b)

def mcm(a,b):
    return abs(a*b) // math.gcd(a,b)

print(mcd(12,18))
print(mcm(12,18))
    \end{lstlisting}
    
    \item \textbf{Agenda de contactos modular:}
    \texttt{agenda.py}:
    \begin{lstlisting}
contactos = {}

def agregar_contacto(nombre, telefono):
    contactos[nombre] = telefono

def buscar_contacto(nombre):
    return contactos.get(nombre, "No encontrado")

def mostrar_contactos():
    return contactos
    \end{lstlisting}
    
    Script principal:
    \begin{lstlisting}
import agenda

agenda.agregar_contacto("Juan", "12345")
print(agenda.buscar_contacto("Juan"))
print(agenda.mostrar_contactos())
    \end{lstlisting}
\end{enumerate}

\end{document}
