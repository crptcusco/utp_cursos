\documentclass[12pt,a4paper]{article}
\usepackage[spanish]{babel}
\usepackage[utf8]{inputenc}
\usepackage[T1]{fontenc}
\usepackage{enumitem}
\usepackage{amsmath}
\usepackage{listings}
\usepackage{xcolor}
\usepackage{geometry}
\geometry{margin=2.5cm}

\lstset{
    language=Python,
    basicstyle=\ttfamily\small,
    keywordstyle=\color{blue},
    stringstyle=\color{red},
    commentstyle=\color{gray},
    showstringspaces=false,
    numbers=left,
    numberstyle=\tiny,
    stepnumber=1,
    frame=single
}

\title{\textbf{Guía de Laboratorio – Sesión 02} \\ Introducción a Python y Tipos de Datos}
\author{Curso de Inteligencia Artificial}
\date{}

\begin{document}
\maketitle

\section*{Objetivo de la sesión}
Al finalizar la sesión, el alumno será capaz de:
\begin{itemize}
    \item Reconocer la sintaxis básica de Python.
    \item Utilizar variables y tipos de datos básicos.
    \item Realizar operaciones aritméticas y lógicas.
    \item Usar las funciones \texttt{print()}, \texttt{input()} y \texttt{type()}.
    \item Convertir entre distintos tipos de datos.
\end{itemize}

\section*{Instrucciones}
Resuelva los siguientes ejercicios en Python. Guarde todos los programas en un archivo por separado, con nombres como \texttt{ejercicio1.py}, \texttt{ejercicio2.py}, etc.

\section*{Ejercicios}

\begin{enumerate}[label=\textbf{Ejercicio \arabic*:}, leftmargin=1.5cm]
    \item \textbf{Hola mundo personalizado:} Solicite el nombre del usuario y muestre el mensaje \texttt{Hola, <nombre>} en pantalla.
    
    \item \textbf{Suma de dos números:} Pida al usuario dos números enteros y muestre su suma, resta, multiplicación y división.
    
    \item \textbf{Conversor de temperaturas:} Convierta una temperatura en grados Celsius a Fahrenheit y Kelvin.
    
    \item \textbf{Tipo de dato:} Lea un valor desde teclado y muestre su valor y su tipo con la función \texttt{type()}.
    
    \item \textbf{Área de un triángulo:} Pida la base y la altura, y calcule el área del triángulo.
    
    \item \textbf{Operaciones con cadenas:} Lea una cadena de texto y:
    \begin{itemize}
        \item Muestre la cantidad de caracteres.
        \item Conviértala a mayúsculas y minúsculas.
        \item Muestre los tres primeros caracteres.
    \end{itemize}
    
    \item \textbf{Conversión de tipos:} Solicite un número decimal y muestre:
    \begin{itemize}
        \item El número original.
        \item Su conversión a entero.
        \item Su conversión a cadena.
    \end{itemize}
    
    \item \textbf{Promedio de notas:} Pida tres notas (números decimales), calcule el promedio y muestre si el estudiante aprobó (nota $\geq$ 13) o reprobó.
    
    \item \textbf{Tiempo en segundos:} Lea horas, minutos y segundos, y conviértalos a la cantidad total de segundos.
    
    \item \textbf{Calculadora simple:} Solicite dos números y un operador (\texttt{+}, \texttt{-}, \texttt{*}, \texttt{/}) y muestre el resultado correspondiente.
\end{enumerate}

\section*{Recomendaciones}
\begin{itemize}
    \item Utilice nombres de variables claros y descriptivos.
    \item Pruebe los programas con distintos valores.
    \item Guarde copias de seguridad de sus archivos.
\end{itemize}

\end{document}
