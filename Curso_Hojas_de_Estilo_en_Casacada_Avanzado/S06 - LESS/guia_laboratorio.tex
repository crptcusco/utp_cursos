\documentclass[12pt, a4paper]{article}
\usepackage[spanish]{babel}
\usepackage[utf8]{inputenc}
\usepackage[T1]{fontenc}
\usepackage{xcolor}
\usepackage{geometry}
\usepackage{listings}
\usepackage[most]{tcolorbox}
\usepackage{enumitem}
\usepackage{hyperref}

% ===== CONFIGURACIÓN =====
\title{\textbf{Guía de Laboratorio: \\ Preprocesador LESS}}
\author{\textcolor{blue}{\textbf{Curso: Hojas de Estilo en Cascada Avanzado}}}
\date{\textbf{Duración: 90 minutos}}

\definecolor{codebg}{RGB}{245,245,245}
\definecolor{blue}{RGB}{41,84,163}
\definecolor{red}{RGB}{192,0,0}
\definecolor{darkblue}{RGB}{0,51,102}
\definecolor{lessgreen}{RGB}{40,160,90}

% ===== ESTILOS =====
\lstdefinestyle{lessstyle}{
    backgroundcolor=\color{codebg},
    basicstyle=\ttfamily\small,
    breaklines=true,
    frame=single,
    rulecolor=\color{gray},
    tabsize=2,
    keywordstyle=\color{blue},
    commentstyle=\color{gray},
    stringstyle=\color{lessgreen},
    literate={á}{{\'a}}1 {é}{{\'e}}1 {í}{{\'i}}1 {ó}{{\'o}}1 {ú}{{\'u}}1
}

\newtcblisting{codebox}[1][]{
    colback=codebg,
    colframe=blue,
    arc=4pt,
    boxrule=0.8pt,
    fonttitle=\bfseries,
    listing only,
    listing options={style=lessstyle},
    #1
}

\newcommand{\ejercicio}[2]{
    \section*{Ejercicio #1: #2}
    \vspace{0.3em}
}

\newcommand{\instrucciones}{
    \noindent\colorbox{blue!10}{
        \parbox{\dimexpr\textwidth-2\fboxsep}{
            \textbf{\textcolor{darkblue}{OBJETIVOS DE LA SESIÓN:}} \\
            - Comprender los fundamentos del preprocesador LESS \\
            - Aprender a utilizar variables, mixins y funciones \\
            - Implementar anidamiento y modularización \\
            - Configurar un entorno de desarrollo para LESS \\
            - Compilar archivos LESS a CSS \\
            \vspace{0.5em}
            \textbf{\textcolor{darkblue}{RECURSOS NECESARIOS:}} \\
            - Editor de código (VS Code recomendado) \\
            - Node.js y npm instalados \\
            - Extensión "LESS" para el editor \\
            - Navegador web actualizado
        }
    }
    \vspace{1em}
}

\begin{document}

\maketitle

\instrucciones

% ===== CONFIGURACIÓN INICIAL =====
\section*{Configuración del Entorno}

\subsection*{Instalación de LESS}
\begin{enumerate}
    \item Verificar que Node.js esté instalado: \texttt{node -v}
    \item Instalar LESS globalmente: \texttt{npm install -g less}
    \item Verificar la instalación: \texttt{lessc -v}
\end{enumerate}

\subsection*{Estructura de Archivos}
Crea la siguiente estructura de carpetas:
\begin{codebox}[title=Estructura de directorios]
proyecto-less/
├── index.html
├── styles/
│   ├── main.less
│   ├── variables.less
│   ├── mixins.less
│   └── componentes/
│       ├── botones.less
│       ├── tarjetas.less
│       └── formularios.less
└── css/
    └── main.css (generado automáticamente)
\end{codebox}

% ===== EJERCICIO 1 =====
\ejercicio{1}{Variables y Operaciones (15 minutos)}

Crea el archivo \texttt{variables.less} y define las siguientes variables:

\begin{codebox}[title=styles/variables.less]
// Colores principales
@color-primario: #3498db;
@color-secundario: #2ecc71;
@color-acento: #e74c3c;

// Escala de grises
@color-texto: #333333;
@color-fondo: #f5f5f5;
@color-borde: #dddddd;

// Tipografía
@fuente-principal: 'Helvetica Neue', Arial, sans-serif;
@tamano-base: 16px;
@linea-altura: 1.5;

// Espaciado
@espaciado: 20px;
@espaciado-medio: @espaciado * 0.75;
@espaciado-pequeno: @espaciado * 0.5;

// Tamaños de pantalla
@movil: 480px;
@tablet: 768px;
@desktop: 1024px;
@desktop-xl: 1200px;
\end{codebox}

Ahora crea el archivo \texttt{main.less} y utiliza las variables:

\begin{codebox}[title=styles/main.less]
// Importar variables
@import "variables";

// Estilos base usando variables
body {
  font-family: @fuente-principal;
  font-size: @tamano-base;
  line-height: @linea-altura;
  color: @color-texto;
  background-color: @color-fondo;
  margin: 0;
  padding: @espaciado;
}

.contenedor {
  max-width: @desktop;
  margin: 0 auto;
  padding: @espaciado;
  
  // Operaciones matemáticas
  @ancho-interno: @desktop - (@espaciado * 2);
  width: @ancho-interno;
}

// Uso de funciones de color
.header {
  background-color: @color-primario;
  color: lighten(@color-primario, 40%);
  padding: @espaciado;
  border-bottom: 2px solid darken(@color-primario, 10%);
}
\end{codebox}

% ===== EJERCICIO 2 =====
\ejercicio{2}{Mixins y Funciones (20 minutos)}

Crea el archivo \texttt{mixins.less} y añade los siguientes mixins:

\begin{codebox}[title=styles/mixins.less]
// Mixin para sombras
.sombra(@x: 0, @y: 2px, @blur: 4px, @color: rgba(0,0,0,0.1)) {
  box-shadow: @x @y @blur @color;
}

// Mixin para bordes redondeados
.border-redondeado(@radio: 4px) {
  border-radius: @radio;
}

// Mixin para transiciones
.transicion(@propiedad: all, @duracion: 0.3s, @timing: ease) {
  transition: @propiedad @duracion @timing;
}

// Mixin para flexbox
.flexbox(@direccion: row, @justify: flex-start, @align: stretch) {
  display: flex;
  flex-direction: @direccion;
  justify-content: @justify;
  align-items: @align;
}

// Mixin responsivo
.responsive(@dispositivo, @contenido) {
  @media (min-width: @dispositivo) {
    @contenido();
  }
}

// Mixin con guard clauses
.texto-contraste(@color) when (lightness(@color) > 50%) {
  color: #000;
}

.texto-contraste(@color) when (lightness(@color) <= 50%) {
  color: #fff;
}
\end{codebox}

Ahora utiliza estos mixins en tu archivo principal:

\begin{codebox}[title=styles/main.less (continuación)]
// Importar mixins
@import "mixins";

// Aplicar mixins
.card {
  .border-redondeado(8px);
  .sombra(0, 4px, 8px, rgba(0,0,0,0.1));
  .transicion(box-shadow, 0.2s, ease-out);
  background-color: #fff;
  padding: @espaciado;
  margin-bottom: @espaciado;
  
  &:hover {
    .sombra(0, 6px, 12px, rgba(0,0,0,0.15));
  }
}

.boton {
  .border-redondeado(4px);
  .transicion();
  display: inline-block;
  padding: @espaciado-pequeno @espaciado;
  border: none;
  cursor: pointer;
  
  &-primario {
    background-color: @color-primario;
    .texto-contraste(@color-primario);
    
    &:hover {
      background-color: darken(@color-primario, 10%);
    }
  }
  
  &-secundario {
    background-color: @color-secundario;
    .texto-contraste(@color-secundario);
    
    &:hover {
      background-color: darken(@color-secundario, 10%);
    }
  }
}
\end{codebox}

% ===== EJERCICIO 3 =====
\ejercicio{3}{Anidamiento y Selectores (15 minutos)}

Aprovecha el anidamiento de LESS para organizar mejor tus estilos:

\begin{codebox}[title=styles/main.less (continuación)]
// Anidamiento de selectores
.menu {
  .flexbox(row, space-between, center);
  list-style: none;
  padding: 0;
  margin: 0 0 @espaciado 0;
  
  // Selector hijo
  > li {
    margin-right: @espaciado;
    
    &:last-child {
      margin-right: 0;
    }
  }
  
  // Selector de elementos anidados
  a {
    text-decoration: none;
    color: @color-texto;
    padding: @espaciado-pequeno;
    .transicion(color);
    
    &:hover {
      color: @color-primario;
    }
  }
  
  // Selector con & para estados especiales
  &.menu-vertical {
    flex-direction: column;
    
    li {
      margin-right: 0;
      margin-bottom: @espaciado-pequeno;
    }
  }
}

// Anidamiento de media queries
.contenedor {
  width: 100%;
  
  .responsive(@tablet, {
    width: 90%;
  });
  
  .responsive(@desktop, {
    width: 80%;
    max-width: 1200px;
  });
}

// Uso de & para modificar elementos
.boton {
  // ... estilos base anteriores ...
  
  &.grande {
    padding: @espaciado @espaciado * 1.5;
    font-size: @tamano-base * 1.25;
  }
  
  &.redondo {
    .border-redondeado(50px);
  }
  
  // Estados del botón
  &:disabled {
    opacity: 0.6;
    cursor: not-allowed;
  }
}
\end{codebox}

% ===== EJERCICIO 4 =====
\ejercicio{4}{Modularización y Organización (20 minutos)}

Crea componentes modulares y organízalos en archivos separados:

\begin{codebox}[title=styles/componentes/botones.less]
// Botones personalizados
.boton-especial {
  .boton;
  position: relative;
  overflow: hidden;
  
  &::after {
    content: '';
    position: absolute;
    top: 0;
    left: -100%;
    width: 100%;
    height: 100%;
    background: linear-gradient(
      90deg,
      transparent,
      rgba(255, 255, 255, 0.2),
      transparent
    );
    .transicion(left);
  }
  
  &:hover::after {
    left: 100%;
  }
}

.grupo-botones {
  .flexbox(row, flex-start, center);
  gap: @espaciado-pequeno;
  
  .boton {
    margin-right: 0;
  }
  
  &.vertical {
    flex-direction: column;
    align-items: stretch;
  }
}
\end{codebox}

\begin{codebox}[title=styles/componentes/tarjetas.less]
// Estilos para tarjetas
.tarjeta {
  .card; // Extiende el mixin .card
  
  &-titulo {
    font-size: @tamano-base * 1.5;
    margin-top: 0;
    margin-bottom: @espaciado-pequeno;
    color: @color-primario;
  }
  
  &-contenido {
    margin-bottom: @espaciado;
    
    &:last-child {
      margin-bottom: 0;
    }
  }
  
  &-acciones {
    .flexbox(row, flex-end, center);
    gap: @espaciado-pequeno;
    margin-top: @espaciado;
    padding-top: @espaciado;
    border-top: 1px solid @color-borde;
  }
  
  // Variantes de tarjetas
  &.sombra-elevada {
    .sombra(0, 8px, 16px, rgba(0,0,0,0.15));
  }
  
  &.sin-borde {
    border: none;
  }
}
\end{codebox}

Ahora actualiza tu archivo principal para importar estos componentes:

\begin{codebox}[title=styles/main.less (continuación)]
// Importar componentes
@import "componentes/botones";
@import "componentes/tarjetas";
@import "componentes/formularios";

// Estilos específicos de la página
.seccion-destacada {
  background-color: lighten(@color-primario, 40%);
  padding: @espaciado * 2;
  margin-bottom: @espaciado * 2;
  .border-redondeado(8px);
  .sombra(0, 4px, 8px, rgba(0,0,0,0.1));
  
  h2 {
    color: @color-primario;
    margin-top: 0;
  }
  
  // Media query anidada
  .responsive(@tablet, {
    padding: @espaciado * 3;
    
    h2 {
      font-size: @tamano-base * 2;
    }
  });
}

// Utilidades de espaciado
.mt-0 { margin-top: 0 !important; }
.mb-0 { margin-bottom: 0 !important; }
.mt-1 { margin-top: @espaciado-pequeno !important; }
.mb-1 { margin-bottom: @espaciado-pequeno !important; }
.mt-2 { margin-top: @espaciado !important; }
.mb-2 { margin-bottom: @espaciado !important; }
.mt-3 { margin-top: @espaciado * 1.5 !important; }
.mb-3 { margin-bottom: @espaciado * 1.5 !important; }
\end{codebox}

% ===== EJERCICIO 5 =====
\ejercicio{5}{Compilación y Optimización (10 minutos)}

Aprende a compilar tus archivos LESS de diferentes maneras:

\begin{codebox}[title=Compilación por línea de comandos]
# Compilar un archivo específico
lessc styles/main.less css/main.css

# Compilar con compresión
lessc styles/main.less css/main.min.css --clean-css

# Compilar con source maps para depuración
lessc styles/main.less css/main.css --source-map

# Compilar con watch (necesita less-watch-compiler)
npm install -g less-watch-compiler
less-watch-compiler styles css

# Usando package.json scripts
{
  "scripts": {
    "build:less": "lessc styles/main.less css/main.css",
    "build:less:min": "lessc styles/main.less css/main.min.css --clean-css",
    "watch:less": "less-watch-compiler styles css"
  }
}
\end{codebox}

Crea un archivo HTML básico para probar tus estilos:

\begin{codebox}[title=index.html]
<!DOCTYPE html>
<html lang="es">
<head>
  <meta charset="UTF-8">
  <meta name="viewport" content="width=device-width, initial-scale=1.0">
  <title>Práctica con LESS</title>
  <link rel="stylesheet" href="css/main.css">
</head>
<body>
  <div class="contenedor">
    <header>
      <ul class="menu">
        <li><a href="#">Inicio</a></li>
        <li><a href="#">Servicios</a></li>
        <li><a href="#">Contacto</a></li>
      </ul>
    </header>
    
    <div class="seccion-destacada">
      <h2>Bienvenido a LESS</h2>
      <p>Esto es una sección destacada con estilos definidos en LESS.</p>
    </div>
    
    <div class="tarjeta">
      <h3 class="tarjeta-titulo">Tarjeta de ejemplo</h3>
      <div class="tarjeta-contenido">
        <p>Esta es una tarjeta con estilos definidos en LESS.</p>
      </div>
      <div class="tarjeta-acciones">
        <button class="boton boton-primario">Aceptar</button>
        <button class="boton boton-secundario">Cancelar</button>
      </div>
    </div>
    
    <div class="grupo-botones">
      <button class="boton boton-primario">Botón 1</button>
      <button class="boton boton-secundario">Botón 2</button>
      <button class="boton-especial">Botón especial</button>
    </div>
  </div>
</body>
</html>
\end{codebox}

% ===== DESAFÍO FINAL =====
\section*{Desafío: Crear un Tema Completo con LESS}

Crea un tema completo utilizando todas las características de LESS que has aprendido:

\begin{enumerate}
  \item Define un sistema de colores con variables para claro/oscuro
  \item Crea mixins para componentes reutilizables
  \item Implementa un sistema de grid responsivo
  \item Crea variantes de componentes usando parámetros en mixins
  \item Organiza el código en archivos modulares
  \item Compila con opciones de optimización
\end{enumerate}

\begin{codebox}[title=Ejemplo de tema con modo claro/oscuro]
// variables.less
// Modo claro
@modo-claro: {
  @fondo: #ffffff;
  @texto: #333333;
  @borde: #dddddd;
  @primario: #3498db;
};

// Modo oscuro
@modo-oscuro: {
  @fondo: #222222;
  @texto: #eeeeee;
  @borde: #444444;
  @primario: #5dade2;
};

// Función para aplicar modo
.aplicar-modo(@modo) {
  @fondo: @modo[@fondo];
  @texto: @modo[@texto];
  @borde: @modo[@borde];
  @primario: @modo[@primario];
}
\end{codebox}

\begin{codebox}[title=Ejemplo de tema con modo claro/oscuro continuación]
// Mixin para modo claro
.modo-claro() {
  .aplicar-modo(@modo-claro);
  
  body {
    background-color: @fondo;
    color: @texto;
  }
  
  .card {
    background-color: lighten(@fondo, 5%);
    border: 1px solid @borde;
  }
}

// Mixin para modo oscuro
.modo-oscuro() {
  .aplicar-modo(@modo-oscuro);
  
  body {
    background-color: @fondo;
    color: @texto;
  }
  
  .card {
    background-color: lighten(@fondo, 5%);
    border: 1px solid @borde;
  }
}

// Aplicar modo según preferencia del usuario
@media (prefers-color-scheme: light) {
  .modo-claro();
}

@media (prefers-color-scheme: dark) {
  .modo-oscuro();
}

// Clase para forzar modo
.modo-claro {
  .modo-claro();
}

.modo-oscuro {
  .modo-oscuro();
}
\end{codebox}

% ===== RECURSOS ADICIONALES =====
\section*{Recursos Adicionales}

\begin{itemize}
  \item \textbf{Documentación oficial:} \url{http://lesscss.org/}
  \item \textbf{Repositorio GitHub:} \url{https://github.com/less/less.js}
  \item \textbf{Playground online:} \url{http://less2css.org/}
  \item \textbf{Extensión para VS Code:} "LESS" by mrmlnc
  \item \textbf{Ejemplos avanzados:} \url{https://lesscss.org/features/}
\end{itemize}

\end{document}