\documentclass[12pt, a4paper]{article}
\usepackage[spanish]{babel}
\usepackage[utf8]{inputenc}
\usepackage[T1]{fontenc}
\usepackage{xcolor}
\usepackage{geometry}
\usepackage{listings}
\usepackage[most]{tcolorbox}

% ===== CONFIGURACIÓN =====
\title{\textbf{Práctica Calificada N° 2 \\ Preprocesadores CSS}}
\author{\textcolor{blue}{\textbf{Curso: Hojas de Estilo en Cascada Avanzado}}}
\date{\textbf{Duración: 90 minutos}}

\definecolor{codebg}{RGB}{245,245,245}
\definecolor{blue}{RGB}{41,84,163}
\definecolor{red}{RGB}{192,0,0}
\definecolor{darkblue}{RGB}{0,51,102}

% ===== ESTILOS =====
\newtcblisting{codebox}[1][]{
    colback=codebg,
    colframe=blue,
    arc=4pt,
    boxrule=0.8pt,
    fonttitle=\bfseries,
    listing only,
    listing options={
        basicstyle=\ttfamily\small,
        breaklines=true,
        tabsize=2,
        literate={á}{{\'a}}1 {é}{{\'e}}1 {í}{{\'i}}1 {ó}{{\'o}}1 {ú}{{\'u}}1
                 {ñ}{{\~n}}1 {¿}{{?`}}1 {¡}{{!`}}1
    },
    #1
}

\newcommand{\pregunta}[2]{
    \section*{Pregunta #1: #2}
    \vspace{0.3em}
}

\newcommand{\instrucciones}{
    \noindent\colorbox{red!10}{
        \parbox{\dimexpr\textwidth-2\fboxsep}{
            \textbf{\textcolor{red}{INSTRUCCIONES:}} \\
            - Tiempo total: 90 minutos \\
            - Resuelve los 4 problemas en orden \\
            - Calificación: 5 puntos cada pregunta (Total: 20 puntos) \\
            - Elige UN preprocesador: SASS, LESS o Stylus \\
            - Entrega: Archivos HTML + CSS del preprocesador elegido
        }
    }
    \vspace{1em}
}

\begin{document}

\maketitle

\instrucciones

% ===== PREGUNTA 1 =====
\pregunta{1}{Variables y Mixins Basicos (5 puntos)}
Crea un sistema de variables y mixins para una pagina web simple:

\textbf{Requisitos:}
\begin{itemize}
\item Define al menos 5 variables (colores, fuentes, espaciado)
\item Crea 2 mixins: uno para sombras y otro para bordes redondeados
\item Aplica las variables y mixins en estilos basicos
\end{itemize}

% \begin{codebox}[title=Ejemplo de estructura]
% // variables
% color-primario = #3498db
% color-secundario = #2ecc71
% espaciado = 16px

% // mixins
% sombra()
%   box-shadow 0 2px 4px rgba(0,0,0,0.1)

% border-redondeado(radio = 4px)
%   border-radius radio

% // uso
% .header
%   background-color color-primario
%   padding espaciado
%   sombra()
% \end{codebox}

% ===== PREGUNTA 2 =====
\pregunta{2}{Componente de Tarjeta (5 puntos)}
Crea un componente de tarjeta reutilizable usando tu preprocesador:

\textbf{Requisitos:}
\begin{itemize}
\item Usa anidamiento para los estilos de la tarjeta
\item Incluye variantes (primaria, secundaria)
\item Aplica los mixins de la pregunta 1
\item Hazla responsive (diferentes tamanos en movil/desktop)
\end{itemize}

% \begin{codebox}[title=HTML de ejemplo]
% <div class="card">
%   <div class="card-header">
%     <h3>Titulo de la tarjeta</h3>
%   </div>
%   <div class="card-body">
%     <p>Contenido de la tarjeta</p>
%   </div>
%   <div class="card-footer">
%     <button class="btn">Accion</button>
%   </div>
% </div>
% \end{codebox}

% ===== PREGUNTA 3 =====
\pregunta{3}{Layout Responsivo (5 puntos)}
Crea un layout simple de 3 columnas que sea responsivo:

\textbf{Requisitos:}
\begin{itemize}
\item 3 columnas en desktop
\item 2 columnas en tablet
\item 1 columna en movil
\item Usar flexbox o grid
\item Incluir un header y footer
\end{itemize}

% \begin{codebox}[title=Estructura HTML]
% <div class="container">
%   <header class="header">Header</header>
%   <main class="main-content">
%     <section class="columna">Columna 1</section>
%     <section class="columna">Columna 2</section>
%     <section class="columna">Columna 3</section>
%   </main>
%   <footer class="footer">Footer</footer>
% </div>
% \end{codebox}

% ===== PREGUNTA 4 =====
\pregunta{4}{Sistema de Temas Claro/Oscuro (5 puntos)}
Implementa un sistema de temas claro/oscuro con un boton de cambio:

\textbf{Requisitos:}
\begin{itemize}
\item Dos temas: claro y oscuro con paletas de colores diferentes
\item Boton que cambie entre temas
\item Los cambios deben aplicar a toda la pagina
\item Transicion suave entre temas
\end{itemize}

% \begin{codebox}[title=HTML para el tema]
% <button id="themeToggle" class="theme-btn">
%   <span class="theme-icon">Modo Oscuro</span>
% </button>
% \end{codebox}

% \begin{codebox}[title=JavaScript basico]
% const themeBtn = document.getElementById('themeToggle');
% themeBtn.addEventListener('click', () => {
%   document.body.classList.toggle('dark-theme');
% });
% \end{codebox}

% ===== RUBRICA =====
\section*{Criterios de Evaluacion - Rubrica (20 puntos)}

\subsection*{Pregunta 1: Variables y Mixins (5 puntos)}
\begin{itemize}
\item \textbf{2pts:} Variables correctamente definidas y usadas
\item \textbf{2pts:} Mixins funcionales y bien aplicados
\item \textbf{1pts:} Sintaxis correcta del preprocesador
\end{itemize}

\subsection*{Pregunta 2: Componente de Tarjeta (5 puntos)}
\begin{itemize}
\item \textbf{2pts:} Anidamiento correcto y organizado
\item \textbf{2pts:} Variantes funcionales y diseno responsive
\item \textbf{1pts:} Integracion con mixins anteriores
\end{itemize}

\subsection*{Pregunta 3: Layout Responsivo (5 puntos)}
\begin{itemize}
\item \textbf{2pts:} Layout funcional en todos los breakpoints
\item \textbf{2pts:} Media queries bien implementadas
\item \textbf{1pts:} Estructura HTML semantica
\end{itemize}

\subsection*{Pregunta 4: Temas Claro/Oscuro (5 puntos)}
\begin{itemize}
\item \textbf{2pts:} Boton funcional que cambia temas
\item \textbf{2pts:} Paletas de colores coherentes para ambos temas
\item \textbf{1pts:} Transicion suave entre temas
\end{itemize}

% \section*{Ejemplo de Solucion - Tema Claro/Oscuro}

% \begin{codebox}[title=Stylus para temas]
% // Variables para tema claro
% bg-claro = #ffffff
% texto-claro = #333333
% primario-claro = #3498db

% // Variables para tema oscuro  
% bg-oscuro = #1a1a1a
% texto-oscuro = #ffffff
% primario-oscuro = #5dade2

% // Mixin para aplicar temas
% tema(bg, texto, primario)
%   background-color bg
%   color texto
  
%   .card
%     background-color lighten(bg, 5%)
%     border 1px solid darken(bg, 10%)
  
%   .btn
%     background-color primario
%     color texto

% // Tema claro (por defecto)
% body
%   tema(bg-claro, texto-claro, primario-claro)
%   transition all 0.3s ease

% // Tema oscuro
% body.dark-theme
%   tema(bg-oscuro, texto-oscuro, primario-oscuro)
% \end{codebox}

\end{document}