\documentclass[12pt, a4paper]{article}
\usepackage[spanish]{babel}
\usepackage[utf8]{inputenc}
\usepackage[T1]{fontenc}
\usepackage{geometry}
\usepackage{xcolor}
\usepackage{listings}
\usepackage[most]{tcolorbox}

% Configuración de colores
\definecolor{codebg}{RGB}{245,245,245}
\definecolor{blue}{RGB}{41,84,163}
\definecolor{darkblue}{RGB}{0,51,102}

% Definir lenguaje CSS para listings
\lstdefinelanguage{CSS}{
    keywords={color,background,margin,padding,font,weight,display,position,top,left,right,bottom,list,style,border,size,white,space,min,max},
    sensitive=true,
    morecomment=[l]{//},
    morecomment=[s]{/*}{*/},
    morestring=[b]',
    morestring=[b]"
}

% Estilo para cajas de código (corregido para UTF-8)
\newtcblisting{codebox}[1][]{
    colback=codebg,
    colframe=blue,
    arc=4pt,
    boxrule=0.8pt,
    fonttitle=\bfseries,
    listing only,
    listing options={
        basicstyle=\ttfamily\small,
        breaklines=true,
        tabsize=2,
        literate={á}{{\'a}}1 {é}{{\'e}}1 {í}{{\'i}}1 {ó}{{\'o}}1 {ú}{{\'u}}1
                  {ñ}{{\~n}}1 {¿}{{?`}}1 {¡}{{!`}}1
    },
    #1
}

% Estilo para enunciados
\newcommand{\enunciado}[1]{
    \vspace{0.5em}
    \noindent\colorbox{blue!10}{
        \parbox{\dimexpr\textwidth-2\fboxsep}{
            \small\sffamily\textcolor{darkblue}{#1}
        }
    }
    \vspace{0.5em}
}

\title{\textbf{Guía de Laboratorio: Flexbox }}
\author{\textcolor{blue}{\textbf{Docente: Carlos R. P. Tovar}}}
\date{}

\begin{document}

\maketitle

\section*{Ejercicio 1: Contenedor Básico}
\enunciado{
    1. Crea un archivo index.html y otro styles.css \\
    2. Define un div.container con 3 elementos div.item \\
    3. Aplica display: flex al contenedor y usa justify-content para centrar los elementos
}

\begin{codebox}[title=HTML]
<!DOCTYPE html>
<html lang="es">
<head>
  <meta charset="UTF-8">
  <title>Flexbox Basico</title>
  <link rel="stylesheet" href="styles.css">
</head>
<body>
  <div class="container">
    <div class="item">1</div>
    <div class="item">2</div>
    <div class="item">3</div>
  </div>
</body>
</html>
\end{codebox}

\begin{codebox}[title=CSS]
.container {
  display: flex;
  justify-content: center;
  background: #f0f0f0;
  padding: 20px;
}

.item {
  background: lightblue;
  padding: 20px;
  margin: 10px;
  width: 80px;
  text-align: center;
}
\end{codebox}

\section*{Ejercicio 3: Dirección y Envoltura}
\enunciado{
    1. Usa flex-direction: column en .container y analiza el efecto \\
    2. Agrega 6 elementos div.item \\
    3. Aplica flex-wrap: wrap para envoltar elementos \\
    4. Experimenta con diferentes alturas de contenedor
}

\begin{codebox}[title=HTML]
<div class="container">
    <div class="item">1</div>
    <div class="item">2</div>
    <div class="item">3</div>
    <div class="item">4</div>
    <div class="item">5</div>
    <div class="item">6</div>
</div>
\end{codebox}

\begin{codebox}[title=CSS]
.container {
    display: flex;
    flex-direction: column;  /* Prueba tambien con row */
    flex-wrap: wrap;
    height: 300px;
    gap: 10px;
    background: #f0f0f0;
    padding: 20px;
}

.item {
    background: lightblue;
    padding: 20px;
    flex: 1 0 80px;
    text-align: center;
}
\end{codebox}

\section*{Ejercicio 4: Tamaño Flexible}
\enunciado{
    1. Modifica .item para usar flex: 1 en lugar de tamaño fijo \\
    2. Experimenta con: \\
    \hspace{1em}• flex-grow: controla el crecimiento relativo \\
    \hspace{1em}• flex-shrink: controla la reducción relativa \\
    \hspace{1em}• flex-basis: establece el tamaño inicial \\
    3. Prueba diferentes combinaciones
}

\begin{codebox}[title=CSS]
/* Opcion 1: Shorthand (crece + encoge proporcionalmente) */
.item {
    flex: 1;  /* Equivale a flex: 1 1 0 */
}

/* Opcion 2: Propiedades detalladas */
.item-1 {
    flex-grow: 1;    /* Factor de crecimiento */
    flex-shrink: 1;  /* Factor de reduccion */
    flex-basis: 100px; /* Tamano base */
}

.item-2 {
    flex-grow: 2;    /* Crecera el doble que item-1 */
    flex-shrink: 0;  /* No se reducira */
    flex-basis: 150px;
}
\end{codebox}

\section*{Ejercicio 5: Galería de Tarjetas (Desafío)}
\enunciado{
    1. Crea una estructura con div.card-container que tenga varias div.card \\
    2. Usa flex-wrap y justify-content: center para distribución responsive \\
    3. Aplica estilos con box-shadow y border-radius \\
    4. Agrega efecto hover para interacción
}

\begin{codebox}[title=HTML]
<div class="card-container">
    <div class="card">
        <h3>Tarjeta 1</h3>
        <p>Contenido de ejemplo</p>
    </div>
    <div class="card">
        <h3>Tarjeta 2</h3>
        <p>Contenido de ejemplo</p>
    </div>
    <!-- Repetir para mas tarjetas -->
</div>
\end{codebox}

\begin{codebox}[title=CSS]
.card-container {
    display: flex;
    flex-wrap: wrap;
    justify-content: center;
    gap: 20px;
    padding: 20px;
    background: #f5f5f5;
}

.card {
    background: white;
    border-radius: 8px;
    box-shadow: 0 2px 6px rgba(0,0,0,0.1);
    width: 200px;
    padding: 20px;
    transition: transform 0.3s ease, box-shadow 0.3s ease;
}

.card:hover {
    transform: translateY(-5px);
    box-shadow: 0 5px 15px rgba(0,0,0,0.2);
}

.card h3 {
    color: #2c3e50;
    margin-top: 0;
}

.card p {
    color: #7f8c8d;
}
\end{codebox}

\end{document}