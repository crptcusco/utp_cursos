\documentclass[12pt, a4paper]{article}
\usepackage[spanish]{babel}
\usepackage[utf8]{inputenc}
\usepackage[T1]{fontenc}
\usepackage{xcolor}
\usepackage{geometry}
\usepackage{listings}
\usepackage[most]{tcolorbox}

% ===== CONFIGURACIÓN =====
\title{\textbf{Práctica Calificada N° 1 }}
\author{\textcolor{blue}{\textbf{Curso: Hojas de Estilo en Cascada Avanzado}}}
\date{\textbf{Unidad 1: Diseño Responsivo}}

\definecolor{codebg}{RGB}{245,245,245}
\definecolor{blue}{RGB}{41,84,163}
\definecolor{red}{RGB}{192,0,0}
\definecolor{darkblue}{RGB}{0,51,102}

% ===== ESTILOS =====
\newtcblisting{codebox}[1][]{
    colback=codebg,
    colframe=blue,
    arc=4pt,
    boxrule=0.8pt,
    fonttitle=\bfseries,
    listing only,
    listing options={
        basicstyle=\ttfamily\small,
        breaklines=true,
        tabsize=2
    },
    #1
}

\newcommand{\pregunta}[2]{
    \section*{Pregunta #1: #2}
    \vspace{0.3em}
}

\newcommand{\instrucciones}{
    \noindent\colorbox{red!10}{
        \parbox{\dimexpr\textwidth-2\fboxsep}{
            \textbf{\textcolor{red}{INSTRUCCIONES:}} \\
            - Tiempo total: 90 minutos \\
            - Resuelve los 4 problemas en orden \\
            - Calificación: 5 puntos cada pregunta \\
            - Se evalúa: funcionalidad, semántica HTML, eficiencia CSS y diseño responsivo \\
            - Entrega: Archivos HTML + CSS comprimidos
        }
    }
    \vspace{1em}
}

\begin{document}

\maketitle

\instrucciones

% ========== VERSIÓN A ==========
% \pregunta{1}{E-commerce Layout con Grid CSS}
% Crea el layout de una tienda online usando Grid CSS que incluya:
% \begin{itemize}
% \item Header con logo, buscador y carrito
% \item Sidebar izquierdo con categorías de productos
% \item Área principal con grid de productos (4 columnas desktop)
% \item Footer con múltiples columnas de información
% \item Layout responsivo que colapse a 1 columna en mobile
% \end{itemize}

% \pregunta{2}{Galería de Imágenes Responsiva con Flexbox}
% Crea una galería de imágenes responsiva usando Flexbox que:
% \begin{itemize}
% \item Muestre 6 imágenes por fila en desktop (1400px+)
% \item Muestre 4 imágenes por fila en tablet (900px-1399px)
% \item Muestre 2 imágenes por fila en mobile (<900px)
% \item Implementar efecto lightbox al hacer clic en imágenes
% \item Usar \textbf{flex-grow} y \textbf{flex-shrink} apropiadamente
% \end{itemize}

% \pregunta{3}{Formulario de Checkout Responsivo}
% Diseña un formulario de checkout para e-commerce que incluya:
% \begin{itemize}
% \item Sección de información de envío (2 columnas desktop)
% \item Sección de método de pago con opciones (tarjeta, PayPal, etc.)
% \item Resumen del pedido con precios y total
% \item Validación HTML5 para todos los campos de pago
% \item Diseño que se apile verticalmente en mobile
% \end{itemize}

% \pregunta{4}{Panel de Administración Híbrido}
% Crea un panel de administración que combine Grid y Flexbox:
% \begin{itemize}
% \item Sidebar colapsable con menú de administración
% \item Header con breadcrumbs y notificaciones
% \item Área principal con estadísticas en grid 2x2
% \item Tabla de datos responsiva que permita scroll horizontal
% \item Modal de edición que use position: fixed
% \end{itemize}

% ========== VERSIÓN B ==========
\pregunta{1}{Blog Responsivo con Grid Areas}
Crea la estructura de un blog moderno usando CSS Grid:
\begin{itemize}
\item Header con navegación y modo oscuro/claro
\item Main con artículos en grid de 2 columnas (desktop)
\item Sidebar derecho con biografía del autor y redes sociales
\item Footer con newsletter y copyright
\item Usar \textbf{grid-template-areas} para definir el layout
\item Navegación debe convertirse en menú hamburguesa en mobile
\end{itemize}

\pregunta{2}{Portafolio de Trabajos con Flexbox}
Diseña una sección de portafolio usando Flexbox que:
\begin{itemize}
\item Muestre proyectos en 3 columnas desktop, 2 tablet, 1 mobile
\item Cada proyecto debe tener imagen, título, categorías y enlace
\item Implementar filtros por categoría usando Flexbox
\item Efecto hover que muestre descripción del proyecto
\item Usar \textbf{flex-order} para destacar proyectos importantes
\end{itemize}

\pregunta{3}{Sistema de Reservas con Validación}
Crea un formulario de reservas para un restaurante que incluya:
\begin{itemize}
\item Selección de fecha y hora con input type="datetime-local"
\item Selección de número de personas con input type="number"
\item Campos para información de contacto con validación
\item Diseño en 2 columnas para desktop, 1 para mobile
\item Mensajes de error personalizados para horarios no disponibles
\end{itemize}

\pregunta{4}{Dashboard de Métricas}
Crea un dashboard de analytics que combine tecnologías:
\begin{itemize}
\item Grid principal para widgets de métricas
\item Flexbox para la navegación superior
% \item Gráficos simulados con CSS (barras, círculos)
\item Sistema de pestañas para diferentes vistas
\item Responsivo: los widgets se apilan verticalmente en mobile
\item Usar \textbf{aspect-ratio} para mantener proporciones
\end{itemize}

% ========== VERSIÓN C ==========
% \pregunta{1}{Sitio de Noticias con Grid Jerárquico}
% Diseña un sitio de noticias con grid complejo:
% \begin{itemize}
% \item Artículo principal destacado (full width)
% \item Grid secundario de 3 columnas para noticias menores
% \item Sidebar con noticias trending
% \item Zona de publicidad que respete el grid
% \item Footer con sitemap en 4 columnas desktop
% \end{itemize}

% \pregunta{2}{Interface de Chat Responsiva}
% Crea una interface de mensajería usando Flexbox:
% \begin{itemize}
% \item Lista de conversaciones (sidebar izquierdo)
% \item Área principal de mensajes con burbujas
% \item Input para escribir mensajes fijo en la parte inferior
% \item Diseño que oculte la lista de conversaciones en mobile
% \item Efectos de transición para suavizar cambios
% \end{itemize}

% \pregunta{3}{Formulario Multistep Responsivo}
% Implementa un formulario multipaso que:
% \begin{itemize}
% \item Use progreso visual con steps (1/4, 2/4, etc.)
% \item Validación entre steps antes de avanzar
% \item Diseño responsivo que mantenga la usabilidad
% \item Botones de navegación (anterior/siguiente) adaptativos
% \item Almacenamiento temporal en localStorage
% \end{itemize}

% \pregunta{4}{Admin Panel con Sistema de Grid}
% Crea un panel de control administrativo:
% \begin{itemize}
% \item Grid system de 12 columnas para organización
% \item Cards que se expanden según importancia
% \item Modal de configuración que use positioning
% \item Tablas responsivas con scroll horizontal
% \item Menú contextual que aparezca al hacer right-click
% \end{itemize}

% ===== RÚBRICA (COMMON TO ALL VERSIONS) =====
\section*{Criterios de Evaluación General}
\begin{itemize}
\item \textbf{Funcionalidad (40\%):} El código funciona como se solicita
\item \textbf{Responsividad (30\%):} Se adapta correctamente a diferentes dispositivos
\item \textbf{Semántica HTML5 (15\%):} Uso apropiado de elementos semánticos
\item \textbf{Calidad de código (15\%):} Código limpio, comentado y organizado
\end{itemize}

\end{document}