\documentclass[12pt, a4paper]{article}
\usepackage[spanish]{babel}
\usepackage[utf8]{inputenc}
\usepackage[T1]{fontenc}
\usepackage{xcolor}
\usepackage{geometry}
\usepackage{listings}
\usepackage[most]{tcolorbox}

% ===== CONFIGURACIÓN =====
\title{\textbf{Guía de Laboratorio 3: Grid Layout}}
\author{\textcolor{blue}{\textbf{Docente: Carlos R. P. Tovar}}}
\date{}

\definecolor{codebg}{RGB}{245,245,245}
\definecolor{blue}{RGB}{41,84,163}
\definecolor{darkblue}{RGB}{0,51,102}

% ===== ESTILOS PARA CÓDIGO =====
\lstdefinelanguage{CSS}{
    keywords={display, grid, grid-template-columns, grid-template-rows, gap, 
              grid-column, grid-row, grid-area, align-items, justify-content},
    sensitive=true,
    morecomment=[l]{//},
    morecomment=[s]{/*}{*/},
}

\newtcblisting{codebox}[1][]{
    colback=codebg,
    colframe=blue,
    arc=4pt,
    boxrule=0.8pt,
    fonttitle=\bfseries,
    listing only,
    listing options={
        basicstyle=\ttfamily\small,
        breaklines=true,
        tabsize=2,
        literate={á}{{\'a}}1 {é}{{\'e}}1 {í}{{\'i}}1 {ó}{{\'o}}1 {ú}{{\'u}}1
    },
    #1
}

\newcommand{\enunciado}[1]{
    \vspace{0.5em}
    \noindent\colorbox{blue!10}{
        \parbox{\dimexpr\textwidth-2\fboxsep}{
            \small\sffamily\textcolor{darkblue}{#1}
        }
    }
    \vspace{0.5em}
}

\begin{document}
\maketitle
\thispagestyle{empty}

% ===== EJERCICIOS =====

\section*{Ejercicio 1: Contenedor Grid Básico}
\enunciado{
    1. Crea un contenedor grid con 3 columnas y 2 filas \\
    2. Usa \texttt{grid-template-columns} y \texttt{grid-template-rows} \\
    3. Añade 6 elementos hijos
}

\begin{codebox}[title=HTML]
<div class="grid-container">
    <div class="item">1</div>
    <div class="item">2</div>
    <!-- ... hasta 6 items -->
</div>
\end{codebox}

\begin{codebox}[title=CSS]
.grid-container {
    display: grid;
    grid-template-columns: 100px 100px 100px;
    grid-template-rows: 50px 50px;
    gap: 10px;
}

.item {
    background: lightblue;
    text-align: center;
    padding: 20px;
}
\end{codebox}

\section*{Ejercicio 2: Unidades Flexibles y \texttt{repeat()}}
\enunciado{
    1. Usa \texttt{repeat()} y unidades \texttt{fr} (fracción) \\
    2. Crea 4 columnas de igual ancho \\
    3. Añade filas con altura mínima de 100px
}

\begin{codebox}[title=CSS]
.grid-container {
    display: grid;
    grid-template-columns: repeat(4, 1fr);
    grid-auto-rows: minmax(100px, auto);
    gap: 15px;
}
\end{codebox}

\section*{Ejercicio 3: Posicionamiento con \texttt{grid-column} y \texttt{grid-row}}
\enunciado{
    1. Haz que el item 1 ocupe 2 columnas \\
    2. Haz que el item 5 ocupe 2 filas \\
    3. Experimenta con \texttt{grid-column-start/end}
}

\begin{codebox}[title=CSS]
.item-1 {
    grid-column: 1 / 3;  /* Ocupa desde línea 1 a 3 */
}

.item-5 {
    grid-row: span 2;    /* Ocupa 2 filas */
}
\end{codebox}

\section*{Ejercicio 4: Áreas de Grid}
\enunciado{
    1. Define áreas con \texttt{grid-template-areas} \\
    2. Asigna elementos a áreas con \texttt{grid-area} \\
    3. Crea un layout de cabecera, contenido y pie
}

\begin{codebox}[title=CSS]
.grid-container {
    display: grid;
    grid-template-areas:
        "header header"
        "sidebar content"
        "footer footer";
    gap: 10px;
}

.header { grid-area: header; }
.sidebar { grid-area: sidebar; }
.content { grid-area: content; }
.footer { grid-area: footer; }
\end{codebox}

\section*{Ejercicio 5: Diseño Responsive (Desafío)}
\enunciado{
    1. Usa media queries para cambiar el grid en móviles \\
    2. Convierte 4 columnas en 2 columnas en pantallas < 600px \\
    3. Haz que el sidebar ocupe toda la anchura en móviles
}

\begin{codebox}[title=CSS]
.grid-container {
    display: grid;
    grid-template-columns: repeat(4, 1fr);
}

@media (max-width: 600px) {
    .grid-container {
        grid-template-columns: repeat(2, 1fr);
    }
    .sidebar {
        grid-column: 1 / -1; /* Ocupa todas las columnas */
    }
}
\end{codebox}

\end{document}