\documentclass[12pt, a4paper]{article}
\usepackage[spanish]{babel}
\usepackage[utf8]{inputenc}
\usepackage[T1]{fontenc}
\usepackage{xcolor}
\usepackage{geometry}
\usepackage{listings}
\usepackage[most]{tcolorbox}



% Configuración básica
\title{\textbf{Guía de Laboratorio 5: Preprocesadores CSS \\ (Introducción a Stylus)}}
\author{\textcolor{blue}{\textbf{Docente: Carlos R. P. Tovar}}}
\date{}

% Colores personalizados
\definecolor{codebg}{RGB}{245,245,245}
\definecolor{blue}{RGB}{41,84,163}
\definecolor{darkblue}{RGB}{0,51,102}

% Definición del lenguaje Stylus (corregido)
\lstdefinelanguage{Stylus}{
    keywords={@import, @require, @media, @keyframes, if, else, for, in, return, mixin, function, use},
    sensitive=false,
    comment=[l]{//},
    morecomment=[s]{/*}{*/},
    string=[s]{"}{"},
    string=[s]{'}{'}
}

% Estilo para cajas de código (optimizado)
\newtcblisting{codebox}[1][]{
    colback=codebg,
    colframe=blue,
    arc=4pt,
    boxrule=0.8pt,
    fonttitle=\bfseries,
    listing only,
    listing options={
        basicstyle=\ttfamily\small,
        breaklines=true,
        tabsize=2,
        literate={á}{{\'a}}1 {é}{{\'e}}1 {í}{{\'i}}1 {ó}{{\'o}}1 {ú}{{\'u}}1
                 {ñ}{{\~n}}1 {¿}{{?`}}1 {¡}{{!`}}1
    },
    #1
}

% Estilo para enunciados
\newcommand{\enunciado}[1]{
    \vspace{0.5em}
    \noindent\colorbox{blue!10}{
        \parbox{\dimexpr\textwidth-2\fboxsep}{
            \small\sffamily\textcolor{darkblue}{#1}
        }
    }
    \vspace{0.5em}
}

\begin{document}
\maketitle
\thispagestyle{empty}

% Ejercicio 1: Sintaxis Básica (corregido)
\section*{Ejercicio 1: Sintaxis Básica}
\enunciado{
    1. Instala Stylus: \texttt{npm install -g stylus} \\
    2. Crea un archivo \texttt{estilos.styl} \\
    3. Compila con: \texttt{stylus estilos.styl -o styles.css}
}

\begin{codebox}[title=Stylus]
/* Variables */
color-primario = #3498db
margen-base = 16px

/* Sintaxis simplificada */
body
    font 100% Arial, sans-serif
    background #f5f5f5
    margin 0
    padding margen-base

/* Anidamiento */
nav
    ul
        list-style none
        padding 0
        li
            display inline-block
            margin-right margen-base
\end{codebox}

% Ejercicio 2: Variables y Operaciones (corregido)
\section*{Ejercicio 2: Variables y Operaciones}
\enunciado{
    1. Define variables para dimensiones \\
    2. Usa operaciones matemáticas \\
    3. Aplica interpolación de variables
}

\begin{codebox}[title=Stylus]
// Cálculos con variables
ancho-base = 960px
ancho-columna = ancho-base / 3
gutter = 20px

.contenedor
    width ancho-base
    margin 0 auto

.columna
    float left
    width "calc(%s - %s)" % (ancho-columna gutter)
    margin-right gutter
    &:last-child
        margin-right 0
\end{codebox}

% Ejercicio 3: Mixins y Funciones (corregido)
\section*{Ejercicio 3: Mixins y Funciones}
\enunciado{
    1. Crea mixins reutilizables \\
    2. Define funciones personalizadas \\
    3. Usa condicionales
}

\begin{codebox}[title=Stylus]
// Mixin para sombras
sombra(nivel = 1)
    if nivel == 1
        box-shadow 0 2px 4px rgba(0,0,0,0.1)
    else if nivel == 2
        box-shadow 0 4px 8px rgba(0,0,0,0.15)

// Función para convertir a em
em(valor, base = 16px)
    (valor / base)em

// Aplicación
.card
    sombra(2)
    border-radius 8px
    padding em(24px)
    font-size em(18px)
\end{codebox}

% Ejercicio 4: Modularización (corregido)
\section*{Ejercicio 4: Modularización}
\enunciado{
    1. Divide el código en módulos \\
    2. Usa @import para combinarlos \\
    3. Compila múltiples archivos
}

\begin{codebox}[title=Stylus]
\begin{verbatim}
src/
|- main.styl
|- variables.styl
|- mixins.styl
|- componentes/
   |- botones.styl
   |- tarjetas.styl
\end{verbatim}

// main.styl
@import 'variables'
@import 'mixins'
@import 'componentes/botones'
@import 'componentes/tarjetas'
\end{codebox}


% Ejercicio 5: Ejemplo
\section*{Ejercicio 5}
\enunciado{
    1. Divide el código en módulos \\
    2. Usa @import para combinarlos \\
    3. Muestra el código Stylus
}

\begin{codebox}[title=CSS]

.contenedor {
  max-width: 1200px;
  margin: 0 auto;
  padding: 20px;
  background-color: #f5f5f5;
}

@media (min-width: 768px) {
  .contenedor {
    padding: 40px;
  }
}

.card {
  box-shadow: 0 3px 10px rgba(0,0,0,0.2);
  border-radius: 8px;
  padding: 20px;
  background-color: white;
}

.card__titulo {
  color: #3498db;
  font-size: 1.5rem;
}

.card__texto {
  color: #333;
  line-height: 1.6;
}

\end{codebox}



\end{document}