\documentclass[12pt, a4paper]{article}
\usepackage[spanish]{babel}
\usepackage[utf8]{inputenc}
\usepackage[T1]{fontenc}
\usepackage{xcolor}
\usepackage{geometry}
\usepackage{listings}
\usepackage[most]{tcolorbox}

% ===== CONFIGURACIÓN =====
\title{\textbf{Guía de Laboratorio: Introducción al Diseño Responsivo}}
\author{\textcolor{blue}{\textbf{Docente: Carlos R. P. Tovar}}}
\date{}

\definecolor{codebg}{RGB}{245,245,245}
\definecolor{blue}{RGB}{41,84,163}
\definecolor{darkblue}{RGB}{0,51,102}

% ===== ESTILOS PARA CÓDIGO =====
\lstdefinelanguage{CSS}{
    keywords={@media, min-width, max-width, width, float, 
              clear, margin, padding, display, position},
    sensitive=true,
    morecomment=[l]{//},
    morecomment=[s]{/*}{*/},
}

\newtcblisting{codebox}[1][]{
    colback=codebg,
    colframe=blue,
    arc=4pt,
    boxrule=0.8pt,
    fonttitle=\bfseries,
    listing only,
    listing options={
        basicstyle=\ttfamily\small,
        breaklines=true,
        tabsize=2,
        literate={á}{{\'a}}1 {é}{{\'e}}1 {í}{{\'i}}1 {ó}{{\'o}}1 {ú}{{\'u}}1
    },
    #1
}

\newcommand{\enunciado}[1]{
    \vspace{0.5em}
    \noindent\colorbox{blue!10}{
        \parbox{\dimexpr\textwidth-2\fboxsep}{
            \small\sffamily\textcolor{darkblue}{#1}
        }
    }
    \vspace{0.5em}
}

\begin{document}
\maketitle
\thispagestyle{empty}



% ===== EJERCICIOS ORIGINALES (optimizados) =====
\section*{Ejercicio 1: Viewport Meta Tag}
\enunciado{
    1. Implementa el meta tag viewport \\
    2. Observa diferencias en dispositivos móviles \\
    3. Prueba con/sin él en Chrome DevTools
}

\begin{codebox}[title=HTML]
<meta name="viewport" content="width=device-width, initial-scale=1.0">
\end{codebox}

\section*{Ejercicio 2: Unidades Relativas}
\enunciado{
    1. Maqueta un contenedor con unidades \% y rem \\
    2. Cambia tamaños de fuente relativamente \\
    3. Ajusta paddings con em
}

\begin{codebox}[title=CSS]
.contenedor {
    width: 90%;
    margin: 0 auto;
    font-size: 1rem;
    padding: 1em;
}

.titulo {
    font-size: 1.5rem;
    margin-bottom: 0.5em;
}
\end{codebox}

% ===== EJERCICIO ADICIONAL: MAQUETACIÓN CON DIVS =====
\section*{Ejercicio 3: Maquetación Responsiva con Divs}
\enunciado{
    1. Crea una estructura básica con divs (header, contenido, sidebar, footer) \\
    2. Usa float para disposición en desktop \\
    3. Con media queries convierte a diseño vertical en móviles \\
    4. Aplica márgenes porcentuales
}

\begin{codebox}[title=HTML]
<div class="contenedor">
    <div class="header">Header</div>
    <div class="main">
        <div class="contenido">Contenido Principal</div>
        <div class="sidebar">Sidebar</div>
    </div>
    <div class="footer">Footer</div>
</div>
\end{codebox}

\begin{codebox}[title=CSS]
/* Estilos base (mobile-first) */
.contenedor {
    width: 100%;
    margin: 0 auto;
}

.header, .footer {
    background: #333;
    color: white;
    padding: 1rem;
    text-align: center;
}

.contenido, .sidebar {
    padding: 1rem;
}

/* Desktop (float layout) */
@media (min-width: 768px) {
    .contenido {
        float: left;
        width: 70%;
        background: #f0f0f0;
    }
    
    .sidebar {
        float: right;
        width: 30%;
        background: #e0e0e0;
    }
    
    .main::after {
        content: "";
        display: table;
        clear: both;
    }
}
\end{codebox}

\section*{Ejercicio 4: Menú Adaptable}
\enunciado{
    1. Crea menú horizontal (desktop) / vertical (mobile) \\
    2. Usa display: inline-block para items \\
    3. Oculta elementos con display: none en móviles
}

\begin{codebox}[title=CSS]
.menu-item {
    display: inline-block;
    margin-right: 1rem;
}

@media (max-width: 600px) {
    .menu-item {
        display: block;
        margin: 0.5rem 0;
    }
    
    .desktop-only {
        display: none;
    }
}
\end{codebox}

\section*{Ejercicio 5: Imágenes Flexibles}
\enunciado{
    1. Haz imágenes adaptables con max-width \\
    2. Usa figure/figcaption para semántica \\
    3. Aplica bordes porcentuales
}

\begin{codebox}[title=CSS]
.img-responsive {
    max-width: 100%;
    height: auto;
    border: 0.5vw solid #ddd;
}

figure {
    margin: 2vh auto;
    text-align: center;
}
\end{codebox}

\end{document}